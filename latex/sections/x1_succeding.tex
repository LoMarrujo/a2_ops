\section{Succeeding with AI}



\subsection{Introduction}
The successful use of AI requires significant human involvement
and insight. AI on its own isn't a substitute for business knowledge,
nor can it improve your business by solely looking at the data and
making recommendations
\footnote{
    \textbf{Warning.}

    The fastest way to fail with AI is for the executive and business
    leaders to think, "AI can solve our problems; we don't need to do
    anything except hire the right tech geeks and unleash them on our
    data," or for the whole data science team to think,
    "Businesspeople take care of the business; we focus on technology."
    Business and technology must work together for success.
}.

To intelligently apply AI, you need special skills and knowledge to
enable you to combine your business domain with it.
\footnote{
    \textbf{Warning.}

    Technical knowledge regarding AI algorithms isn't sufficient to
    get business results using AI.
}.

What you need to know to run and get good results from an AI-based
project. It's assumed that you can run a general technical project.

A team that pays attention to how to link technology with the business
problem will beat a team that consists of (slightly) better specialists
in a limited area of technology or business. If you aren't taking a
genuine interestin what the other side does, this book isn't for you.
\footnote{
    \textbf{Warning.}

    If you're a data scientist, this book will be of value to you, as
    long as you're interested in how to achieve business results with
    AI. This book expects the reader who has a business background to
    be willing to learn the basics of how AI works. Similarly, if
    you're a leadership-focused data scientist, you're expected to be
    willing to learn about the business side of the equation.
}.


\subsection{AI and the Age of Implementation}
The ways AI is best used in academia and industry are different.

Successfully applying AI to business problems is not research in
the traditional academic sense. Unless you're a researcher at a
university (or in an industry-supported R&D) who's getting paid
for conducting research itself, additional research results are
difficult to monetize. You need the best practices for
implementing AI projects.

Academia focuses on discovering new principles, industry is more
concerned with implementing existing knowledge. What we're
doing today is applying known principles of AI to concrete
business problems. If we're in the Age of Implementation, what
we need most are practices for the best application of AI to
new areas of the business and this is the case given the context
of an AI project in business and industry today
\footnote{
    In almost all industry projects, it's dangerous for both
    your project and your career to depend on the need to make
    new scientific discoveries to be able to deliver the project.
}.


\subsection{How do you make money with AI?}
Simply analyzing data with AI is not enough it's essential to
have a clear plan for how to use the results to drive business
value. He suggests that this plan should be developed at the
beginning of the project [and refined if necessary during
the course of the project], rather than after the analysis is
complete
\footnote{
    The main insight to remember is that using AI to analyze
    data doesn't make money; properly reacting to the results
    of the correct analysis does.
}.

It is important to remember that it is not always about money.
Not all organizations are interested just in making profit.
If you're working for a charity or non-governmental nonprofit
organization (NGO), you have nonmonetary goals, such as the
number of people you help. If you're a nonprofit devoted to
helping people, then substitute an appropriate, quantifiable
metric for the word “money.” How many people did you help? To
what degree did you help them?


\subsection{What matters for your project to succeed?}
Let us consider the best practices of time and project
management. Not every technology you'd use on your project
is equally important for success. Nor is every part of your
project equally important. Your time should be focused on 
those parts of your project that matter the most for success.

An AI project's leadership has to make many choices, and it's
easy to get lost in them. Those choices are in many different
technical areas—areas like what infrastructure to use, how to
handle big data, whether to use cloud or on-premise solutions,
and which AI algorithms to use. Within each one of those
categories are many additional choices you'd have to make
\footnote{
    \begin{itemize}
        \item Clound, on-premise?
        \item Do we need to handle big data?
        \item Which algorithms/models?
        \item Which metrics for evaluation?
        \item What monitoring and security infraestrcture do
        you need?
    \end{itemize}
}.

Although you must choose your infrastructure correctly
(on a do not sink level), the primary determinant of the AI
project's success lies in how well you connect the answer to
your research question with a specific business action.
Once you're confident in that connection, then you can focus
your attention on infrastructure and making good infrastructural
choices.

Don't get bogged down in technical details early on.
When embarking on an AI project, the highest priority for
management attention must be on linking a research question
to specific business action.

Infrastructure is an enabler, not the focus
\footnote{
    Let's embrace the analogy that data is the new oil. If you
    need a new oilfield, you'd put your greatest focus into
    finding oil and understanding what an oilfield looks like.
    You most certainly wouldn't start by buying the best oil
    drilling equipment in the world, with the drilling location
    as an afterthought. The same thing applies to data.
    concentrate on finding oil, not on the drilling equipment.
}. Infrastructure is only critical in large enterprises where
broad data science adoption requires it. As a project leader,
you must be careful that you don't intermix the enabler with
the value you're building. Your primary focus should be on
value. You should never allow yourself to be in the situation
in which you're discussing infrastructure but have no concrete
idea of what specific business questions you intend to use AI
to answer
\footnote{
    There is one exception to the "Don't start with the
    infrastructure” rule. If the enterprise is large, you may
    need an infrastructure that supports as wide a range as
    possible of technologies used in AI today. Only the largest
    companies are in this position, and, for them, my advice is
    that they build infrastructure in parallel with the
    definition and development of data science use cases that
    need to be supported.
}.


\subsection{Machine learning from 10,000 feet}
Let's got through the roles that humans will need to fill for a
project using AI to succeed.

Although ML can process data and optimize specific metrics, it
lacks comprehension of the meaning behind the data
\footnote{
    ML is not magical—it's a mathematical tool that operates on
    numbers to optimize a metric. The only goal an ML algorithm
    has is to find an optimal value of a given evaluation
    metric, all the while not knowing or caring about why it's
    optimizing that particular metric.

    Informally, ML is the application of some mathematical
    algorithm that requires data input in a certain format,
    produces an answer in a predefined output format, and
    doesn't provide any other guarantee except that it will
    minimize some number that we call the evaluation metric.

    ML certainly is not magical. The trick of using ML is how
    you use them. The real value comes from how humans apply
    and interpret the results.
}.

The use of ML involves three parts
\footnote{
    If you're an executive, one way to think about ML is as a
    black box that operates on numbers. Who then is responsible
    for bringing intelligence to that ignorant black box?
    Humans, through the proper formulation (that's the job of
    data scientists) and application of the appropriate
    evaluation metric that measures something relevant to your
    business (that's your job as an executive, in cooperation
    with the data scientist).
}: 
\begin{enumerate}
    \item Formulation (framing the business problem in terms of
    input/output data),
    
    \item Optimization (how the ML algorithm finds optimal
    solutions with respect to an evaluation metric), and
    
    \item Evaluation (measuring the success of the optimization
    using relevant metrics).
\end{enumerate}
It's up to you as the user of the algorithm to understand which
evaluation metric is best and what the best format is for your
data.

If you're a data scientist, you're aware of what ML is. But
here's a detail that's easy to miss: because the only guarantee
provided by the algorithm is that it will optimize the
evaluation metrics, those metrics must have a business meaning
\footnote{
    Those metrics can't just be something you've seen in an
    academic paper, and certainly not something you don't know
    how to relate to anything concrete in your business.

    When a data science team is choosing the evaluation metrics,
    they must do so in close cooperation with the executive
    team. Here's the scary thing—a lot of ML projects
    use evaluation metrics that aren't clearly tied to business
    results.
}.

At best, most of the technical evaluation metrics are merely
correlated with but not identical to the business result you're
trying to get. At worst, those evaluation metrics that your data
scientist can't even explain in business terms are in the will
not help your business category
\footnote{
    If you're an executive, a good habit for you to form is to
    speak up when you see an acronym in a business presentation
    that you don't know (for example, RMSE). Similarly, with ML,
    demand an explanation in the form of, “Don't tell me the
    mathematical definition of that metric; tell me how to
    relate the value of that metric to something in my business.”
}.

If your evaluation metric isn't tied to your business result,
you're using a black box (the ML algorithm) to generate a random
number (a value of the evaluation metric) and then trying to
figure out how to run your business based on that random number.
Good luck!


\subsection{Start by understanding the possible business actions}
Focused human involvement and attention makes or breaks an AI
project.

You don't make money simply by knowing the answer to a business
question—you make money when you take action. The number of good,
effective actions you can take to affect the physical world is
relatively small. Many analyses you can perform will yield
results that are not actionable.

Don't ask a question if you can't imagine what you'd do with the
answer. You should start an AI project by asking, "What actions
can I take, and what analysis do I need to do to inform those
actions?"

Once you know the business actions you can take, you should use
these actions to drive the analysis, not vice versa.

In every AI project, these are the two most important ideas to
remember:
\begin{enumerate}
    \item Action is where you make profit; analysis without action
    is just cost. You make money when you perform an appropriate
    business action, not when some analysis is completed
    \footnote{
        Analysis can be an enabler of making profits, but from an
        accounting perspective, analysis is a cost. Analysis stops
        being a cost and becomes an investment only when it can
        help you take good business actions.
    }.

    \item Focus on the whole system, not on its individual parts.
    Your customers will never see the individual parts of the
    system.
\end{enumerate}
[Not all projects are profits based, some are also 
risk-minimization or regulatory-requirements based.]


\subsection{Don't fish for "something in the data"}
Teams that fail to start a project by focusing on which business
actions they can take will routinely encounter two failure modes:
\begin{itemize}
    \item copying something that “worked for another organization”.
    
    Copying parts of a system that worked for someone else doesn't
    work, because that copying then confines you to the decisions
    someone else made for their context, whereas your project
    should deliver a whole system customized to solve your
    business problem.
    
    \item use AI to look for “something in the data”.
    What's the chance that you'd actually be able to execute on
    some random action that the analysis just divined? How much
    money do you want to spend on the off chance of getting an
    idea unrelated to your day-to-day business operations?
    \footnote{
        Sometimes you can execute on an unexpected insight. Your
        success average will be higher if the studies you perform
        are related to areas of your business where you can take
        action based on the possible results of your study.
    }
\end{itemize}
Neither of those approaches works well in practice.


\subsection{AI finds correlations, not causes!}
AI as practiced in business and industry today finds correlations,
not causes
\footnote{
    This is important to recognize because newcomers to AI, after
    seeing AI answer a complicated question, sometimes
    anthropomorphize AI and attribute to it the ability to know
    the reason behind the answer when it's often just guessing.
}.
AI might predict an event, but few AI algorithms would impute any
cause: they don't say why it is, just that it is.

When you're thinking about what you need AI to do for your
business, keep in mind the differences between the human use of
causal models and how AIs work:
\begin{itemize}
    \item AI models find patterns, but it does not find causes
    automatically.

    \item If an explanation why an AI algorithm made some decision,
    carefully considerion must be paid to how to translate the
    algorithm's behavior in a way that you and your audience can
    understand.

    \item Most AI algorithms used today, especially in the big
    data domain, have a limited ability to guide change.
    That ability is limited to situations in which you have
    data for both where you are now and where you want to go.
\end{itemize}

Always tread carefully when you think about using AI to impute
causes or to guide change. There are significant limits to what is
known even in academic settings about inferring causality in
complex scenarios.


\subsection{Business results must be measurable!}
For an AI project to succeed, you need to be able to measure the
results of the data science project in the context of its business
impact. That measurement needs to be quantifiable
\footnote{
    AI and ML algorithms can't use gut feeling metrics as
    feedback on how well the project is doing, so someone needs
    to define a quantitative metric.
}.
Before you can measure how your data scientists would impact your
business, you need to be able to measure your business as it is
today.

Once you have those business metrics, you should have a way to
tie those metrics to the technical evaluation metric that the ML
algorithm uses. Depending on the methodology you're using to run
your business, business metrics could be readily available, or
you might have to develop them yourself
\footnote{
    You may not have a direct business metric at your department
    level to which you can direct your data science project. If
    that's the case and you don't already have a metric to use,
    define one yourself.
}.

The complexity associated with recognizing and defining good
business metrics is exactly why I recommend that you develop a
thorough understanding of the operations of your business
\footnote{
    A good business metric yields a number that's directly related
    to some business result. Such a metric is actionable. In the
    case of a recommendation engine, a good business metric could
    be expected increase in profit. Any technical metric that has
    no clear meaning in the context of business is obviously not
    a good business metric. In the absence of quantitative
    business metrics, you're running a project based on gut
    feeling.
}.


\subsection{What is CLUE?}

