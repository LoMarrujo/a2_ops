% Fundamentals of Data Engineering
\section{Designing Good Data Architecture}
Successful data engineering is built upon rock-solid data architecture.
Good data architecture provides seamless capabilities across every
step of the data lifecycle and undercurrent
\footnote{
    Before giving a definition of data architecture,
    it is useful to understand the context of Enterprise Architecture
    (EA).

    The objective of EA is to help organizations align their business
    goals with the IT infrastructure and processes.

    EA is a business function concerned with the design and execution
    of systems to support the change in the enterprise, achieved
    by flexible and reversible decisions reached through careful
    evaluation of trade-offs.

    Architects are not simply mapping out IT processes and vaguely
    looking toward a distant, utopian future; they actively solve
    business problems and create new opportunities.

    Technical solutions exist not for their own sake but in support
    of business goals.
    
    Architects identify problems in the current state (poor data
    quality, scalability limits, money-losing lines of business),
    define desired future states (agile data-quality improvement,
    scalable cloud data solutions, improved business processes),
    and realize initiatives through execution of small, concrete steps.
}.





\subsection{What Is Data Architecture?}
There is no general consensus on a satisfactory definition of 
Data Architecture. But, there are pragmatic and domain-specific,
working definitions for data architecture that work for companies of
vastly different scales, business processes, and needs.

A good working definition of Data Architecture is as follows.
Data Architecture is a subset of enterprise architecture, inheriting
its properties: processes, strategy, change management, and evaluating
trade-offs.

Here are a couple of definitions of data architecture that influence
our definition:
\begin{itemize}
    \item \textbf{TOGAF.}
    
    \noindent
    \begin{verbatim}
        A description of the structure and interaction of the
        enterprise's major types and sources of data,
        logical data assets, physical data assets,
        and data management resources.
    \end{verbatim}


    \item \textbf{DAMA DMBOK.}
    
    \noindent
    \begin{verbatim}
        Identifying the data needs of the enterprise (regardless of
        structure) and designing and  maintaining the master
        blueprints to meet those needs. Using master blueprints to
        guide data integration, control data assets, and align data
        investments with business strategy.
    \end{verbatim}
\end{itemize}

Data architecture is the design of systems to support the evolving
data needs of an enterprise, achieved by flexible and reversible
decisions reached through a careful evaluation of trade-offs.



\subsection*{How does data architecture fit into data engineering?}
Data engineering architecture is the systems and frameworks
that make up the key sections of the data engineering lifecycle.
We'll use data architecture interchangeably with data engineering
architecture.

Other aspects of data architecture that you should be aware of are
operational and technical.
\begin{itemize}
    \item \textbf{Operational architecture} encompasses the functional
    requirements  of what needs to happen related to people, processes,
    and technology
    \footnote{
        e.g. What business processes does the data serve?
        How does the organization manage data quality?
        What is the latency requirement from when the data is
        produced to when it becomes available to query?
    }.

    \item \textbf{Technical architecture} outlines how data is
    ingested, stored, transformed, and served along the data
    engineering lifecycle
    \footnote{
        e.g. How will you move 10 TB of data every hour from a source
        database to your data lake?
    }.
\end{itemize}
In summary, operational architecture describes what needs to be done,
and technical architecture details how it will happen.


\subsection*{"Good" Data Architecture}
Now that we have a working definition of data architecture, let's
cover the elements of “good” data architecture.

What do we mean by “good” data architecture?
To paraphrase an old cliche, you know good when you see it.
Good data architecture serves business requirements with a common,
widely reusable set of building blocks while maintaining flexibility
and making appropriate trade-offs
\footnote{
    In contrast, bad architecture is authoritarian and tries to cram
    a bunch of one-size-fits-all decisions into a big ball of mud.

    % http://www.laputan.org/mud/
    Note that a big ball of mud architecture is a casually, even
    haphazardly, structured system. Its organization, if one can
    call it that, is dictated more by expediency than design.
    Yet, its enduring popularity cannot merely be indicative of a
    general disregard for architecture. 
}.

Good data architecture is flexible and easily maintainable. Businesses
and their use cases for data are always evolving. Agility is the
foundation for good data architecture; it acknowledges that the world
is fluid.




% Fundamentals of Data Engineering, 2022
\subsection{Principles of Good Data Architecture}

\begin{enumerate}
    \item \textbf{Choose Common Components Wisely.}
    
    \noindent
    Common components can be anything that has broad applicability
    within an organization. Common components include object storage,
    version-control systems, observability, monitoring and
    orchestration systems, and processing engines.

    Choose common components and practices that can be used widely
    across an organization. When architects choose well and lead
    effectively, common components become a fabric facilitating team
    collaboration and breaking down silos.

    Common components must support robust permissions
    and security to enable sharing of assets among teams while preventing unauthorized
    access.


    \item \textbf{Plan for Failure}
    
    \noindent
    Modern hardware is highly robust and durable. Even so, any
    hardware component will fail, given enough time. To build highly
    robust data systems, you must consider failures in your designs.
    Here are a few key terms for evaluating failure scenarios:
    \begin{itemize}
        \item \textbf{Availability} is the percentage of time an IT service
        or component is in an operable state.
        
        \item \textbf{Reliability}  the system's probability of
        meeting defined standards in performing its intended
        function during a specified interval.
        
        \item \textbf{Recovery time objective} (RTO) is the maximum
        acceptable time for a service or system outage.
        
        The RTO is generally set by determining the business impact of
        an outage. An RTO of one day might be fine for an internal
        reporting system. A website outage of just five minutes could
        have a significant adverse business impact on an online
        retailer.
                
        \item \textbf{Recovery point objective} 
        The acceptable state after recovery. In data systems, data is
        often lost during an  outage. In this setting, the recovery
        point objective (RPO) refers to the maximum  acceptable data
        loss.
    \end{itemize}
    Consider acceptable A,R, RTO, and RPO in designing for failure.
    These will guide their architecture decisions as they assess
    possible failure scenarios.


    \item \textbf{Architect for Scalability}
    
    \noindent
    Scalability in data systems encompasses two main capabilities:
    \begin{enumerate}
        \item scalable systems can scale up to handle significant
        quantities of data.

        \item scalable systems can scale down.
    \end{enumerate}
    An elastic system can scale dynamically in response to load,
    ideally in an automated fashion. Some scalable systems can also
    scale to zero: they shut down completely when not in use.

    Note that deploying inappropriate scaling strategies can result
    in overcomplicated systems and high costs.


    \item \textbf{Architecture Is Leadership}
    
    \noindent
    Data architects are
    \footnote{
        Data architects should be highly technically competent but
        delegate most individual contributor work to others. Strong
        leadership skills combined with high technical competence are
        rare and extremely valuable.
    }
    responsible for technology decisions and architecture descriptions
    and disseminating these choices through effective leadership and
    training.

    Note that leadership does not imply a command-and-control approach
    to technology.

    As a data engineer, you should practice architecture leadership
    and seek mentorship from architects. Eventually, you may well
    occupy the architect role yourself.


    \item \textbf{Always Be Architecting}
    
    \noindent
    Data architects constantly design new and exciting things in
    response to changes in business and technology. They don't serve
    in their role simply to maintain the existing state
    
    Per the EABOK, an architect's job is to develop deep knowledge of
    the baseline architecture (current state), develop a target
    architecture, and map out a sequencing plan to determine
    priorities and the order of architecture changes.

    We add that modern architecture should not be command-and-control
    or waterfall but collaborative and agile. The data architect
    maintains a target architecture and sequencing plans that change
    over time.


    \item \textbf{Build Loosely Coupled Systems}
    
    \noindent
    For software architecture, a loosely coupled system has the
    following properties:
    \begin{enumerate}
        \item Systems are broken into many small components.
        \item These systems interface with other services through
        abstraction layers
        \footnote{
            These abstraction layers hide and protect internal details
            of the service, such as a database backend or internal
            classes and method calls.
        },
        such as a messaging bus or an API.
        \item As a consequence of property 2, internal changes to a
        system component don't require changes in other parts
        \footnote{
            Details of code updates are hidden behind stable APIs.
            Each piece can evolve and improve separately.
        }.
        \item As a consequence of property 3, there is no waterfall,
        global release cycle for the whole system. Instead, each
        component is updated separately as changes and improvements
        are made.
    \end{enumerate}
    Notice that we are talking about technical systems. Let's
    translate these technical characteristics into organizational
    characteristics:
    \begin{itemize}
        \item Many small teams engineer a large, complex system. Each
        team is tasked with  engineering, maintaining, and improving
        some system components.

        \item These teams publish the abstract details of their
        components to other teams via API definitions, message
        schemas, etc. Teams work together through loosely coupled
        communication.

        \item Each team can rapidly evolve and improve its component
        independently of the work of other teams.

        \item Teams release continuously during regular working hours
        to make code changes and test them.
    \end{itemize}


    \item \textbf{Make Reversible Decisions}
    
    \noindent    
    The data landscape is changing rapidly. Today's hot technology or
    stack is tomorrow's afterthought. Popular opinion shifts quickly.
    You should aim for reversible decisions
    \footnote{
        As Fowler wrote, “One of an architect's most important tasks
        is to remove architecture by finding ways to eliminate
        irreversibility in software designs.
    },
    as these tend to simplify your architecture and keep it agile.


    \item \textbf{Prioritize Security}
    
    \noindent
    Every data engineer must assume responsibility for the security
    of the systems they build and maintain. We focus now on two main
    ideas: zero-trust security and the shared responsibility security
    model.

    There is a responsibility to design a security model for the
    applications and data and leverage cloud capabilities to realize
    this model.


    \item \textbf{Embrace FinOps}
    
    \noindent
    FinOps (Financial Operations) is a framework that helps
    organizations manage cloud financials effectively by fostering
    collaboration between engineering, finance, and business teams.
    It focuses on optimizing cloud spending while maintaining
    performance and innovation.
\end{enumerate}



%GPT
\subsection{Key Components of Data Architecture}

\begin{itemize}
    \item \textbf{Data Sources}: Systems that generate or capture
    data, such as databases, applications, IoT devices, and
    external APIs.

    \item \textbf{Data Storage}: Solutions like data lakes, data
    warehouses, and databases where data is stored based on its
    type, structure, and usage.

    \item \textbf{Data Integration}: Processes and tools (ETL/ELT)
    that consolidate data from multiple sources into a unified view.

    \item \textbf{Data Modeling}: Designing data structures
    (conceptual, logical, and physical models) to represent and
    organize data efficiently.

    \item \textbf{Data Governance}: Policies and standards for data
    quality, privacy, security, and compliance.

    \item \textbf{Data Processing}: Workflows and pipelines that
    transform raw data into usable formats for analytics and
    applications.

    \item \textbf{Data Access}: Mechanisms for querying and
    accessing data securely, including APIs and data catalogs.
\end{itemize}

\section*{Why Is Data Architecture Important?}

\begin{itemize}
    \item \textbf{Improves Decision-Making}: Ensures reliable and timely data access.
    \item \textbf{Scalability}: Facilitates handling growing data volumes efficiently.
    \item \textbf{Cost Optimization}: Helps manage storage and processing resources effectively.
    \item \textbf{Data Security}: Implements safeguards for data privacy and compliance.
\end{itemize}
























% Fundamentals of Data Engineering, 2022
\subsection{Major Architecture Concepts}


\subsection*{Domains and Services}


\subsection*{Distributed Systems, Scalability, and Designing for Failure}


\subsection*{Tight Versus Loose Coupling: Tiers, Monoliths, and Microservices}


\subsection*{User Access: Single Versus Multitenant}


\subsection*{Event-Driven Architecture}


\subsection*{Brownfield Versus Greenfield Projects}







\subsection{Examples and Types of Data Architecture}






\subsection{Who's Involved with Designing a Data Architecture?}
Data architecture isn't designed in a vacuum. Ideally, a data engineer
will work alongside a dedicated data architect
\footnote{
    Bigger companies may still employ data architects, but those
    architects will need to be heavily in tune and current with the
    state of technology and data.

    Gone are the days of ivory tower data architecture. In the past,
    architecture was largely orthogonal to engineering. We expect this
    distinction will disappear as data engineering, and engineering in
    general, quickly evolves, becoming more agile, with less
    separation between engineering and architecture.
}.
However, if a company
is small or low in its level of data maturity, a data engineer might
work double duty as an architect. Because data architecture is an
undercurrent of the data engineering lifecycle, a data engineer
should understand “good” architecture and the various types of data
architecture.



