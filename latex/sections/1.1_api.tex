\section{APIs}

As ..., 
Understanding Application Programming Interfaces (APIs) is crucial
for integrating machine learning models, automating workflows, and
accessing data from external sources.

Here's everything you need to know:
\begin{itemize}
    \item \textbf{What is an API?}

    \noinent
    An API is a set of rules that allows applications and computers
    to communicate with each other
    \footnote{
        For example, consider building a stock's prirce prediction
        model based on past prices, you'll at least need a way to
        get the data and some kind of output and if another process
        consumes the output of the model to present it in a user's 
        interface.
        
        It's a good idea to have a document or standard that described
        how to build such a connection or interface. A computer system
        that meets this standard is said to \textbf{implement} or 
        \textbf{expose} an API.
    }.
    It abstracts complex operations and provides a simple interface
    to interact with a system.



    \item \textbf{Types of APIs}

    \noindent
    The most common APIs for ML are:
    \begin{itemize}
        \item \textbf{RESTful} (a.k.a. REST) API is an interface that
        two computer systems use to exchange information securely over
        the internet.
        
        Representational State Transfer (REST) is a software architecture
        that imposes conditions on how an API should work. REST was
        initially created as a guideline to manage communication on a
        complex network like the internet. 

        RESTful APIs support this information exchange because they
        follow secure, reliable, and efficient software communication
        standards.

        The most important concepts are:
        \begin{itemize}
            \item \textbf{Endpoint}: URLs where API services are accesed.
            
            \item \textbf{HTTP methods}:
            \begin{itemize}
                \item \textbf{get}: retrieves data.
                \item \textbf{post}: sends data.
                \item \textbf{put}: update data.
                \item \textbf{delete}: remove data.
            \end{itemize}
            
            \item \textbf{Request Headers}: Metadata like authentication
            tokens.

            \item \textbf{Request Body}: json payload sent in POST/PUT requests.            

            \item \textbf{HResponse Codes:}:
            \begin{itemize}
                \item \textbf{200 OK}: success.
                \item \textbf{400 BAD REQUEST}: client error.
                \item \textbf{401 Unauthorized}: authentication required.
                \item \textbf{500 Internal Server Erro}: API-side issue.
            \end{itemize}
        \end{itemize}

        \item GraphQL.
        

        \item SOAP.


        \item gRPC APIs.
        

        \item WebSocket.

    \end{itemize}


    \item \textbf{Authentication and Security}

    \noindent
    APIs often expose sensitive data and functionality, making security
    a critical aspect. Without proper authentication, your API could be
    exploited, leading to data breaches or unauthorized access.
    Understanding authentication methods ensures your API remains secure
    while maintaining usability.

    Some of the most common authentication methods are:
    \begin{itemize}
        \item \textbf{API Keys}: Basic token-based access.
        \item \textbf{OAuth 2.0}: Secure, used for user authentication
        (e.g., Google, AWS).
        \item \textbf{JSON Web Token (JWT)}: Token-based authentication
        with encryption.
    \end{itemize}
    When developing APIs, keep these best practices for security in 
    mind:
    \begin{itemize}
        \item Never expose API keys in public code.
        \item Use environment variables for storing credentials.
        \item Implement rate limiting to avoid abuse.
        \item se HTTPS to encrypt communication.
    \end{itemize}

    \item \textbf{API Monitoring & Logging}.
    \begin{itemize}
        \item Use loguru or logging to track API requests.
        \item Tools like Prometheus + Grafana for performance tracking.
    \end{itemize}

    \item \textbf{Synchronous and Asynchronous Requests}.

    A synchronous request blocks the client until operation completes.
    In such case, javascript engine of the browser is blocked. On the
    other hand, an asynchronous request doesn't block the client i.e.
    browser is responsive. At that time, user can perform another
    operations also. In such case, javascript engine of the browser
    is not blocked.
\end{itemize}



\subsection{Working with APIs in Data Science.}
The most  common use  cases of APIs for Data Science are:
\begin{enumerate}
    \item Fetching data
    (e.g. Twitter's API, Yahoo Finance, Geolocation APIs).
    \item Deploying Machine Learning Models via APIs
    (e.g. FastAPI, Flask, AWS Lambda, Google Cloud Functions).
    \item Automating Workflows
    (e.g. Airflow, Prefect, or Dagster).
    \item Data Streaming
    (e.g. WebSockets or Kafka for real-time data).
\end{enumerate}