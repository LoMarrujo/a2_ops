\section{Queries, Modeling, and Transformation}



\subsection{What Is a Query?}
A query allows you to retrieve and act on data.

Queries are fundamental for retrieving and acting on
data (especially tabular and semi-structured) in data
engineering, science, and analysis. They represent the
"Read" operation in CRUD but can also handle Create,
Update, and Delete.

Types of Query Language Commands:
\begin{itemize}
    \item \textbf{DDL (Data Definition Language):}
    Defines database objects (e.g., CREATE, DROP
    databases/tables).

    \item \textbf{DML (Data Manipulation Language):}
    Manipulates data within objects (e.g., SELECT,
    INSERT, UPDATE, DELETE records).

    \item \textbf{DCL (Data Control Language):}
    Controls access permissions (e.g., GRANT, REVOKE).

    \item \textbf{TCL (Transaction Control Language):}
    Manages transactions (e.g., COMMIT, ROLLBACK).
\end{itemize}


When a query is executed it goes through:
\begin{itemize}
    \item Compilation
    (parsing, semantic checks, permission verification).
    
    \item Conversion to efficient bytecode.

    \item Optimization by the query optimizer
    (reordering steps for efficiency).

    \item Execution to produce results.

    \item The Query Optimizer is crucial for performance,
    analyzing factors like joins, indexes, and data scan
    size to find the least expensive execution plan.
\end{itemize}
The Query Optimizer is crucial for performance, analyzing
factors like joins, indexes, and data scan size to find
the least expensive execution plan.


Improving Query Performance:
\begin{itemize}
    \item \textbf{Optimize Joins:} rejoin data for frequent
    analytics, simplify join logic, use indexes, be wary of
    "row explosion" (many-to-many matches creating excessive
    rows), and use CTEs for readability and potential
    performance gains over subqueries/temp tables.
    
    \item \textbf{Use Explain Plans:} Understand the
    optimizer's plan (EXPLAIN command) and monitor resource
    usage (disk, memory, network, time, data scanned/shuffled,
    concurrency).

    \item \textbf{Avoid Full Table Scans:} Select only needed
    columns/rows. Use pruning techniques like partitioning,
    clustering, and indexes.

    \item \textbf{Understand Commits & Consistency:} Know how
    your database handles transactions (e.g., ACID compliance,
    locking, write concurrency). Different databases (like
    PostgreSQL, BigQuery, MongoDB) have varying models
    affecting consistency and performance.

    \item \textbf{Vacuum Dead Records:} Periodically remove
    outdated records left by transactions to free space,
    improve query plans, and clean indexes. Handling varies
    significantly by database (e.g., auto-vacuum, time-travel
    retention).
 
    \item \textbf{Leverage Caching:} Utilize cached query
    results for frequently run queries to improve speed and
    reduce costs.
\end{itemize}


%! queries on streaming data










\subsection{Data Modeling}
Data modeling is the deliberate process of structuring data
to coherently reflect an organization's real-world processes,
business logic, definitions, and workflows.

Key points:
\begin{itemize}
    \item \textbf{Often Overlooked but Critical:}
    Data modeling is frequently neglected, yet it is an
    essential process for creating useful data systems within
    an organization.

    \item \textbf{Definition:} It involves deliberately
    structuring data to coherently reflect an organization's
    real-world processes, business logic, definitions, and
    workflows.

    \item \textbf{Essential for Business Value:} Well-constructed
    data models are crucial for making data useful and translating
    it into impactful business outcomes and decisions (e.g., by
    consistently defining key terms like "customer").

    \item \textbf{Historical Context \& Resurgence:} While
    techniques like normalization have existed for decades,
    modeling became less fashionable with early data lakes and NoSQL
    systems (sometimes leading to "data swamps"). Its importance is
    now being recognized again, driven by needs for data governance
    and quality.

    \item \textbf{Consequences of Neglect:} Failing to model data
    canlead to confusing, incoherent, redundant, or incorrect
    data ("data swamps").

    \item \textbf{Future Needs:} While
    traditional modeling remains vital, new paradigms might be
    required to fully address the needs of modern streaming data
    and machine learning applications.
\end{itemize}




Data modeling involves progressing from abstract ideas to concrete
database implementation through three main stages:
\begin{enumerate}
    \item \textbf{Conceptual Model:}
    \begin{itemize}
        \item Focuses on \textbf{business logic and rules}.
        \item Describes the system's data entities (e.g., customers,
        products, orders) and their relationships.
        \item Often visualized using \textbf{Entity-Relationship
        (ER) diagrams}.
        \item Answers \textit{what} data the system contains.
    \end{itemize}

    \item \textbf{Logical Model:}
    \begin{itemize}
        \item Details \textit{how} the conceptual model will be
        practically implemented.
        \item Adds significantly more detail, including
        \textbf{data types} for fields and the mapping of
        \textbf{primary and foreign keys}.
        \item Acts as a bridge between the conceptual idea and
        the physical storage.
    \end{itemize}

    \item \textbf{Physical Model:}
    \begin{itemize}
        \item Defines the specific implementation
        \textit{within a chosen database system}.
        \item Includes details like specific database names,
        schema names, table names, and technical
        configurations.
        \item Answers \textit{how} the data will be stored in a
        specific technology.
    \end{itemize}
\end{enumerate}

\begin{itemize}
    \item \textbf{Stakeholder Involvement:} Successful data
    modeling requires active participation from business
    stakeholders from the start to capture correct definitions
    and business goals, ensuring the data produced is high-quality
    and useful.

    \item \textbf{Data Grain:}
    \begin{itemize}
        \item Refers to the level of detail or resolution
        at which data is stored (e.g., individual transaction
        vs. daily summary).
        \item It is strongly recommended to model data at the
        
        \textbf{lowest possible (most granular) level}. This
        allows for flexibility, as granular data can always be
        aggregated, but detail lost through aggregation cannot
        be recovered.
    \end{itemize}
\end{itemize}



\subsection*{Normalization}
Normalization is a database modeling practice aimed at reducing
data redundancy and improving data integrity (applying the DRY
- Don't Repeat Yourself - principle).
\begin{itemize}
    \item Introduced by E.F. Codd with goals like eliminating
    insertion/update/deletion anomalies and making the model
    stable and informative
    \footnote{
        It's worth spending a moment to unpack a couple of terms we just threw at you. A
        unique primary key is a single field or set of multiple fields that uniquely determines
        rows in the table. Each key value occurs at most once; otherwise, a value would map
        to multiple rows in the table. Thus, every other value in a row is dependent on (can
        be determined from) the key. A partial dependency occurs when a subset of fields in
        a composite key can be used to determine a nonkey column of the table. A transitive
        dependency occurs when a nonkey field depends on another nonkey field.
    }.

    \item Involves progressing through sequential Normal Forms (NF):
    \begin{itemize}
        \item \textbf{Denormalized:} No normalization. Nested and
        redundant data is allowed
        \footnote{
            Imagine that inside a table, in a value there is a nested 
            table. This is an example of a denormalized table.
        }.

        \item \textbf{First normal form (1NF):} Normalization
        (to first normal form) is a process where attributes with
        non-simple domains are extracted to separate stand-alone
        relations.

        The extracted relations are amended with foreign keys
        referring to the primary key of the relation which contained
        it. The process can be applied recursively to non-simple
        domains nested in multiple levels
        \footnote{
            As in the example of a denormalized table, we would
            de-nest the table in possible several tables with
            foreign keys with the objective that each row
            represents individual records (non-nested).
        }.
        
        \item \textbf{Second normal form (2NF):}
        
        A relation is in 2NF if:
        \begin{enumerate}
            \item It is 1NF.
            
            \item It does not have a key that is functionally 
            dependent
            \footnote{
                Attribute B is functionally dependent on
                attribute A (written as A -> B) if each value
                of A is associated with exactly one value of B.
            }
            on any proper subset of any candidate key of the 
            relation.
        \end{enumerate}
        
        In other words, a relation (or table) is in 2NF if:
        \begin{enumerate}
            \item It is in 1NF and has a single attribute unique
            identifier (UID).

            \item It is in 1NF and has a multi-attribute UID,
            and every regular attribute (not part of the UID)
            is dependent on all attributes in the multi-attribute
            UID, not just one attribute (or part) of the UID.
        \end{enumerate}

        To make a 1NF relation a 2NF relation, remove the
        functionally dependent attributes in the partial
        dependencies of the first normal form relation, and place
        those partial dependency dependent attributes in a relation
        where their corresponding determinant attributes are an
        entire candidate key.
        
        Achieve 1NF and remove partial dependencies (where non-key
        attributes depend on only part of a composite primary key).

        \item \textbf{Third normal form (3NF):} Achieve 2NF and
        remove transitive dependencies (where non-key attributes
        depend on other non-key attributes). 
    \end{itemize}
    \item Higher forms exist but are less common.

    \item Often achieved by splitting tables to isolate related data.
    
    \item While 3NF is a common standard, the optimal degree depends
    on the use case; some denormalization might be chosen for
    performance reasons (e.g., in OLAP systems).
\end{itemize}









\subsection*{Techniques for Modeling Batch Analytical Data}
he fundamental goal is to transform raw data (which can be structured or
semi-structured) from various sources into a structured format (rows and
columns) suitable for analysis within a data lake or data warehouse.
The primary techniques discussed are Inmon, Kimball, and Data Vault,
though wide denormalized tables and even avoiding explicit modeling are
also options. Often, these techniques can be combined.


\begin{enumerate}
    \item \textbf{Inmon Methodology (Top-Down, Normalized EDW)}
    
    \begin{itemize}
        \item \textbf{Philosophy:} Create a centralized, integrated,
        subject-oriented, nonvolatile, and time-variant Enterprise
        Data Warehouse (EDW) as the single source of truth.

        Department-specific analysis is served via data marts, which
        are created from the central EDW. These data marts might be
        denormalized (e.g., using star schemas) for easier querying
        by specific departments (like sales or marketing).

        \item \textbf{Structure:} The core EDW is highly normalized,
        typically adhering to Third Normal Form (3NF). This minimizes
        data redundancy and aims to closely mirror the normalized
        structure of source systems.

        \item \textbf{Process:} Relies heavily on ETL (Extract,
        Transform, Load) before data enters the EDW. Data from
        disparate sources is cleaned, integrated, and normalized
        during this process.

        \item \textbf{Pros:} Strong data integration, minimal data
        redundancy, clear "single source of truth" for the enterprise.

        \item \textbf{Cons:} Can be time-consuming and complex to
        set up initially (requires building the comprehensive EDW first),
        potentially less flexible to rapid changes in source systems or
        business requirements compared to other methods.
    \end{itemize}
    

    \item \textbf{Kimball Methodology (Bottom-Up, Dimensional Model)}
    \begin{itemize}
        \item \textbf{Philosophy:} Focuses on delivering business value
        quickly by building the data warehouse iteratively from
        department-specific data marts. The data warehouse effectively
        becomes a collection of integrated data marts.

        \item \textbf{Structure:} Based on the\textbf{dimensional model},
        most commonly the \textbf{Star Schema}. Two fundamental concepts:
        \begin{itemize}
            \item \textbf{Fact Tables:}
            Contain quantitative measurements or metrics of business events
            (e.g., sales amount, quantity ordered). They are typically narrow
            (few columns), long (many rows), append-only, and should be at the
            lowest possible level of detail (granularity). They contain foreign
            keys to dimension tables.

            \item \textbf{Dimension Tables:}
            Contain descriptive attributes that provide context to the facts
            (e.g., customer details, product information, date attributes).
            They are often denormalized (accepting some redundancy), wide
            (many columns), and relatively short (fewer rows than facts).


            \item \textbf{Slowly Changing Dimensions (SCDs):}
            Techniques to handle changes in dimension attributes over time
            (e.g., a customer changing address). Common types include:
            Type 1: Overwrite the old value (no history);
            Type 2: Add a new row for the new value, keeping the
            old row with validity dates (most common for history tracking);
            Type 3: Add a new column to store the previous value.

            \item \textbf{Conformed Dimensions:}
            Dimensions shared across multiple star schemas/fact tables to
            ensure consistency.

        \end{itemize}

        \item \textbf{Pros:} Faster time-to-value for specific business areas,
        generally easier for business users to understand and query
        (star schema), often optimized for read performance due to fewer
        joins than 3NF.

        \item \textbf{Cons:} Can lead to data redundancy, potential for
        inconsistent data definitions if conformed dimensions aren't managed
        carefully, integration across the enterprise might require more
        planning.
    \end{itemize}




    \item \textbf{Data Vault Methodology (Hybrid, Auditability & Flexibility)}

    \begin{itemize}
        \item \textbf{Philosophy:} Designed for agility, scalability, and
        auditability, especially in environments with rapidly changing
        source systems. It separates structural information (business
        keys and relationships) from descriptive attributes. Loads raw
        data with minimal transformation.

        \item \textbf{Structure:} Consists of three main table types:
        \begin{enumerate}
            \item \textbf{Hubs:} Store unique business keys
            (e.g., Customer ID, Product ID) and metadata like load
            date and record source. Insert-only.

            \item \textbf{Links:} Track relationships between business keys
            (Hubs). Capture many-to-many relationships. Insert-only.

            \item \textbf{Satellites:} Contain the descriptive attributes
            (context) related to Hubs or Links, including historical changes.
            They track data lineage and changes over time, connected via the
            Hub/Link key and a load date.
        \end{enumerate}

        \item \textbf{Process:} Focuses on insert-only loading, preserving
        the raw history from source systems. Business logic and transformations
        are typically applied after the Data Vault, often when creating
        downstream data marts or views for reporting
        (which might use a Kimball-style model).

        \item \textbf{Pros:} Highly adaptable to changes in source systems,
        provides excellent auditability and traceability, supports parallel
        data loading well.

        \item \textbf{Cons:} Can result in a large number of tables and
        require more complex joins for final analytical queries
        (often necessitating an information mart layer on top),
        can have a steeper learning curve.

    \end{itemize}

    \item \textbf{Wide Denormalized Tables}
    \begin{itemize}
        \item \textbf{Philosophy:} Leverage cheap cloud storage and
        the performance of modern columnar databases by putting
        potentially hundreds or thousands of fields into a single,
        highly denormalized table.

        \item \textbf{Structure:} One very wide table, often sparse
        (many null values). Can include nested data types (like JSON).
        Organized around one or more key columns defining the grain.

        \item \textbf{Context:} Made practical by:
        \begin{itemize}
            \item Cheap cloud storage cost makes storing
            redundant/sparse data viable.
            \item Efficiently store and query wide, sparse tables
            because they only read necessary columns, and nulls take
            minimal space/query time. Schema evolution (adding columns)
            is often simpler.
        \end{itemize}

        \item \textbf{Pros:} Simple to load data, can offer very fast
        query performance for certain queries by eliminating joins,
        flexible schema evolution.

        \item \textbf{Cons:} Loss of clear business logic structure
        within the model, potential redundancy, update operations
        can be inefficient, can become unwieldy without governance.
    \end{itemize}

    \item \textbf{Wide Denormalized Tables}
    \begin{itemize}
        \item \textbf{Philosophy:} Avoid building a separate
        analytical data structure; query operational databases directly.
    
        \item \textbf{Pros:} Quickest way to get initial answers
        for simple questions.

        \item \textbf{Cons:} High risk of impacting operational
        system performance, difficult to ensure query consistency
        and accuracy, lacks historical data integration, no single
        view of the business, not scalable for complex analytics.
        Usually a temporary or very limited solution.
    \end{itemize}

\end{enumerate}

Key Considerations When Choosing:
\begin{itemize}
    \item \textbf{Business Needs:}  Are you focused on enterprise-wide
    consistency (Inmon) or faster departmental reporting (Kimball)?

    \item \textbf{Source System Volatility:} How often do your source
    systems change (Data Vault excels here)?

    \item \textbf{Reporting/Query Complexity:} How complex are the
    typical analytical queries? (Kimball often simpler for end-users).

    \item \textbf{Need for Auditability/Lineage:} Is tracing data history
    critical (Data Vault)?
    
    \item \textbf{Development Resources/Timelines:} How quickly do you need
    results? (Kimball/Wide Tables might be faster initially).
    
    \item \textbf{Technology Stack:} Cloud columnar databases make Wide
    Tables more feasible.
    
    \item \textbf{Team Skills:} Familiarity with the different modeling paradigms.
    
\end{itemize}
Ultimately, the choice often involves trade-offs between integration,
flexibility, query performance, development speed, and maintainability.

Modern approaches frequently blend techniques, such as using Data Vault
for raw data integration and history, feeding Kimball-style marts for reporting.











\subsection*{Modeling Streaming Data}












\subsection{Transformations}
In data engineering, transformations
\footnote{
    The net result of transforming data is the ability to unify
    and integrate data. Once data is transformed, the data can
    be viewed as a single entity. But without transforming data,
    you cannot have a unified view of data across the
    organization.

    —Bill Inmon
}
refer to the process of changing data from one format, structure,
or value state to another. It's a crucial step in making raw data
usable, consistent, and valuable for analysis, reporting,
machine learning, or storage in a target system 
(like a data warehouse or data lakehouse).

A transformation differs from a query. A query retrieves the data
from various sources based on filtering and join logic.
A transformation persists the results for consumption by
additional transformations or queries. These results may be stored
ephemerally or permanently.

Why are Transformations necessary? Because raw data collected from
various sources often comes in different shapes and sizes and can be:
\begin{itemize}
    \item \textbf{Inconsistent:} Different systems may represent the
    same thing differently (e.g. "USA", "U.S", etc.).

    \item \textbf{Incomplete:} Missing values or fields.

    \item \textbf{Incorrect:} Errors, typos, or invalid entries.
    
    \item \textbf{Poorly Structured:} Not organized in a way that's easy
    to query or analyze.

    \item \textbf{In the Wrong Format:} Dates stored as text, numbers
    as strings, incompatible units, etc.

    \item \textbf{Contains Sensitive Information:} Needs masking or
    anonymization.

    \item \textbf{Too Granular or Not Granular Enough:} Needs aggregation
    or splitting.
\end{itemize}

Transformations critically rely on one of the major undercurrents
in this book: orchestration. Orchestration combines many discrete
operations, such as intermediate transformations, that store data
temporarily or permanently for consumption by downstream
transformations or serving. Increasingly, transformation pipelines
span not only multiple tables and datasets but also multiple systems.



\subsection*{Where Do Transformations Happen?}
Transformations are a core part of data pipelines, most notably in:
\begin{enumerate}
    \item \textbf{ETL (Extract, Transform, Load):}
    Data is extracted from sources, transformed in a staging area
    or in-memory, and then loaded into the target system
    (like a data warehouse or data mart).    

    \item \textbf{ELT (Extract, Load, Transform):} Data is
    extracted and loaded directly into the target system
    (often a data lake or modern data warehouse),
    and transformations are performed within the target
    system using its processing power.
\end{enumerate}

In modern data lakehouse architectures, which combine features of data
lakes and warehouses often using object storage, the strict distinction
between ETL and ELT becomes fuzzy. Concepts like data federation and
virtualization further blur the lines of where data resides and is
processed.

Data teams should choose the most appropriate technique (ETL or ELT)
for each specific data pipeline based on its requirements.



\subsection*{Common Types of Data Transformations.}
Here are some specific actions that fall under the umbrella of transformations:
\begin{itemize}
    \item \textbf{Cleaning:}
    \begin{itemize}
        \item Handling null or missing values (imputing, removing).
        \item Correcting errors and typos.
        \item Filtering (e.g. removing duplicate records).
        \item Standardizing formats (e.g., dates to 'YYYY-MM-DD', state names
        to abbreviations).
    \end{itemize}
    
    \item \textbf{Structuring / Reshaping:}
    \begin{itemize}
        \item Joining data from multiple tables or sources.
        
        \item Splitting columns
        (e.g., separating full name into first and last name).

        \item Pivoting (turning rows into columns) or
        Unpivoting (turning columns into rows).

        \item Changing data types (e.g., string to integer,
        string to date).
    \end{itemize}

    \item \textbf{Enrichment:}
    \begin{itemize}
        \item Adding new calculated fields (e.g., calculating age
        from date of birth, profit margin from sales and cost).

        \item Appending data from external sources (e.g.,
        adding demographic information based on zip codes).
    \end{itemize}

    \item \textbf{Aggregation:}
    \begin{itemize}
        \item Summarizing data (e.g., calculating total sales
        per region, average customer spend).   

        \item Grouping data based on certain attributes.
    \end{itemize}

    \item \textbf{Anonymization / Masking:}
    Removing or obscuring personally identifiable information (PII)
    or other sensitive data.
\end{itemize}



\subsection*{Batch Transformations}
Batch Transformation processes operate on discrete, finite chunks
of data, unlike streaming transformations which handle continuous
data flow. Batch jobs typically run on a fixed schedule
(e.g., hourly, daily) to update datasets for reporting, analytics,
or machine learning models.

Batch processing offers several advantages that make it a preferred
choice for handling large volumes of data:
\begin{itemize}
    \item \textbf{Scalability:} Batch processing is highly scalable,
    making it ideal for organizations dealing with enormous datasets.
    As data volume grows, batch processing can handle the load
    efficiently by processing data in manageable chunks.
    
    \item \textbf{Cost-Effectiveness:} By processing data in scheduled
    or periodic batches, organizations can optimize resource
    utilization, reducing costs associated with real-time processing
    systems.

    \item \textbf{Reduced Complexity:} Batch processing systems are
    often simpler to design, implement, and maintain compared to
    real-time systems. This simplicity makes it easier to troubleshoot
    and debug potential issues.

    \item \textbf{Fault Tolerance:} Batch processing can be made
    fault-tolerant by implementing mechanisms to handle errors
    gracefully. If a job fails during processing, it can be rerun
    without affecting the entire system.

    \item \textbf{Consistency:} Since batch processing operates on
    fixed datasets, it ensures consistent results over time. This
    consistency is essential for tasks like financial reporting or
    data reconciliation.
\end{itemize}

%   python example
% import pandas as pd

% # Function to process each batch
% def process_batch(batch_data):

% # Perform data transformations here
% processed_data = batch_data.apply(lambda x: x * 2)  # Example transformation: doubling the values return processed_data

% # Read the CSV file in chunks
% chunk_size = 100000  # Adjust the chunk size as per your system's memory constraints
% csv_file = "large_data.csv"  # Replace with your CSV file's path
% output_file = "processed_data.csv"  # Replace with the desired output file path

% # Read the CSV file in batches and process each batch
% for chunk in pd.read_csv(csv_file, chunksize=chunk_size):
%     processed_chunk = process_batch(chunk)
%     processed_chunk.to_csv(output_file, mode="a", header=False, index=False)



\subsection*{Distributed Joins:}
When joining large datasets that are spread across multiple
machines (nodes) in a cluster (common in systems like Spark,
MapReduce, BigQuery, Snowflake), the logical join operation
must be broken down into smaller joins executed in
parallel on these nodes.

Two Main Distributed Join Strategies: 
\begin{itemize}
    \item \textbf{Broadcast Join:} This is used when one of
    the tables being joined is small enough to fit entirely
    in the memory of a single node. The system copies
    ("broadcasts") this small table to every node
    participating in the join. Each node then joins its
    local partition of the large table with the complete
    small table. This is highly efficient as it avoids
    large-scale data movement (shuffling) across the network.
    Query optimizers often try to enable broadcast joins by
    filtering data early or reordering join operations.

    \item \textbf{Shuffle Hash Join:} This strategy is
    employed when both tables are too large to be broadcast.
    Data from both tables is redistributed ("shuffled")
    across the cluster nodes based on a hash calculated
    from the join key(s). This ensures that all rows with
    the same join key from both tables end up on the same
    node. Each node then performs a hash join on the data
    partitions it received. This method is significantly
    more resource-intensive (network I/O, CPU, memory) due
    to the costly shuffle step.
\end{itemize}
While Broadcast and Shuffle Hash are common, other distributed
join strategies exist, such as Sort-Merge Join
(which often also involves a shuffle phase to sort data by join
key before merging). The query optimizer within the distributed
system chooses the strategy based on table sizes, statistics,
data distribution, and system configuration.

In cloud environments where resources are paid for, the higher
resource consumption (CPU, memory, network) of shuffle joins
compared to broadcast joins translates directly to higher
operational costs. Efficient join strategies save money.


\subsection*{SQL and non-SQL for transformations.}
The distinction between SQL-based and non-SQL transformation
tools is becoming less meaningful, as SQL is now a
"first-class citizen" in most major big data and streaming
frameworks (Spark SQL, Hive, Flink SQL, Kafka SQL, Beam SQL).
The more relevant distinction is now between SQL-only tools and
tools supporting general-purpose programming languages
(like Python/Scala with Spark).

Although SQL is declarative (you define the desired end state)
rather than procedural (defining step-by-step instructions),
it's incorrect to assume it cannot build complex pipelines.
Techniques like Common Table Expressions (CTEs), SQL scripting,
and integration with orchestration tools allow for complex
workflow construction.

When to prefer code (e.g. Python/Spark/Scala) over SQL
\footnote{
    The choice between SQL and code can also be influenced
    by team skills. SQL is generally more accessible to a
    broader range of roles (including analysts), while
    complex transformations in code require data
    engineering or software development expertise.
}:
\begin{enumerate}
    \item \textbf{Complexity:} If the transformation logic
    is very difficult or awkward to express in SQL
    (e.g., complex text manipulation like word stemming).

    \item \textbf{Readability/Maintainability:}
    If the SQL implementation becomes too convoluted and
    hard to understand or modify.
    
    \item \textbf{Reusability/Libraries:}
    When custom, reusable code libraries are needed.
    SQL's native support for this is limited
    (UDFs have drawbacks, though tools like dbt improve
    SQL reusability).
\end{enumerate}

There is an Optimization Trade-off: Code-based frameworks
like Spark give developers fine-grained control but also
place the burden of optimization squarely on them.
SQL engines, conversely, have built-in query optimizers
(like Spark's Catalyst) that handle much of this
automatically.


\subsection*{Update Patterns}
Updating persisted data from transformations is crucial but
historically challenging, especially in data lakes which
originally lacked robust update mechanisms. Modern data
lakehouses address this, driven by practical needs and
compliance requirements (like GDPR deletions).

Update Patterns:
\begin{itemize}
    \item \textbf{Truncate and Reload:} The simplest
    pattern, delete all existing data in the target table
    and reload it completely with the latest transformed
    data.

    \item \textbf{Insert Only:} Add new records without
    altering old ones. The current state is often
    determined at query time by finding the latest record
    per key (computationally expensive) or managed using
    views/materialized views. Frequent single-row inserts
    are an anti-pattern for columnar databases;
    batch/micro-batch loading is preferred.

    \item \textbf{Delete:}  Removing data. Can be hard
    delete (permanent removal, needed for
    compliance/performance) or soft delete
    (marking records as deleted). An insert deletion
    variant adds a new record flagged as deleted,
    fitting insert-only models but adding query complexity.
    Deletes are generally expensive in columnar/file-based
    systems.

    \item \textbf{Upsert/Merge:} Updates existing records
    if a key matches, inserts if no match (Upsert).
    Merge adds the capability to delete records based on
    conditions. This pattern, common in row-based databases,
    is often problematic and resource-intensive in columnar
    systems due to underlying mechanics.
\end{itemize}

Updates/deletes/merges are costly due to Copy-on-Write.
Merging large batches can be performant, but frequent,
small merges (e.g., near real-time CDC) often overwhelm
columnar systems. Batching updates (e.g., hourly) is
generally advisable. Effective partitioning and
clustering strategies are vital for performance.
Specialized features in systems like BigQuery
(streaming buffer) and Druid (storage tiers)
offer near real-time capabilities as alternatives.



\subsection*{Schema Updates}
Schema Evolution
\footnote{
    Schema Evolution usually refers to planned,
    managed changes to a schema. Schema Drift
    typically refers to unexpected or unmanaged
    changes, often from external sources. Both 
    necessitate robust handling strategies.
}
is inevitable. Data sources change,
leading to unavoidable updates in data schemas
(adding fields, changing types, etc.).

While columnar databases make data updates
more complex than row-based systems, they
generally make schema updates (adding,
deleting, renaming columns) technologically
easier. Despite the tech, managing schema
changes practically is difficult. 




\subsection*{Data Wrangling}
Data wrangling is described as the essential,
often difficult, batch transformation process of
cleaning messy, malformed source data into a structured,
usable format. It's highlighted as a traditional pain
point for data engineers, illustrated by the complex,
iterative process of handling problematic EDI data
(initial ingestion struggles, text parsing, anomaly
discovery, gradual refinement).

Data Wrangling Tools aim to simplify and automate
wrangling. Positioned as "IDEs for malformed data"
rather than just "no-code" tools, they offer benefits
like: 
\begin{itemize}
    \item Visual interfaces for data profiling
    (showing types, stats, distributions, outliers,
    nulls).

    \item Step-by-step recipe building for cleaning
    actions (fixing types, splitting columns, joining).

    \item Execution of these recipes at scale, often
    leveraging backends like Apache Spark.

    \item Iterative refinement based on errors
    encountered during execution.

    \item Freeing up engineers from tedious parsing
    and potentially enabling analysts.

    \item Engineers are encouraged to explore options
    from cloud providers and third parties.
\end{itemize}
Data wrangling is fundamentally tied to improving Data Quality.
The profiling features act as initial quality checks, and the
transformation steps directly address identified quality issues
(incorrect types, outliers, nulls, inconsistencies).
Formal Data Quality rules can often be implemented as part of
the wrangling recipe.


\subsection*{Business logic and derived data}
A primary function of data transformation is to encode
complex business logic, typically in batch processes,
resulting in derived data (data computed from other data).

Calculating specialized business metrics involves
intricate logic, potentially accounting for factors
like fraud estimation, order flags, and various
marketing cost attribution models.

An interesting Proposed Solution involves centralizing
business logic definitions within a Metrics Layer.
This layer allows analysts to build reports using
predefined, governed metrics. The layer itself
generates the complex underlying queries, sending
them to the data warehouse for computation, thus
separating logic definition from consumption.


\subsection*{MapReduce}







\subsection*{Materialized Views, Federation, and Query Virtualization}


\subsection*{Views}
A database view
\footnote{
    Views can be used for security (by restricting data
    access), simplification (hiding complex joins/logic),
    and presenting specific data states (e.g. a 
    deduplicated view). 

    The main drawback of views is the computation defined
    in the view runs every time the view is queried,
    which can be slow for complex views.
}
is a subset of a database and is based
on a query that runs on one or more database tables.
In practice, a view is just a query that references
other tables. When queried, the database merges the
view's logic with the user's query, optimizes,
and executes it dynamically. Views don't store data
themselves. 

A \textbf{Materialized View (MV)} addresses the
performance drawback of standard views by
precomputing and physically storing the result set
of the view's query. 

The database manages the process of updating/refreshing
the stored results when the underlying source tables
change. Querying an MV reads the precomputed data,
making it much faster. MVs act as database-managed
transformation steps.

Staleness vs. Performance Trade-off: Standard views
always reflect the current data but can be slow.
MVs offer fast query performance but might present
slightly stale data depending on their refresh
frequency and mechanism
\footnote{
    e.g.  ON COMMIT or ON DEMAND
}.
This is a key design trade-off.


\subsection*{Federation}
Federated Queries are a database feature allowing an
OLAP system (like a data warehouse) to directly query
and retrieve data from external sources, such as
object storage (e.g., S3) or other databases
(e.g., MySQL, PostgreSQL), 
and get the result back as a temporary table.

The main benefit of Federated Queries is that
it improves the integration between data
warehouses and data lake environments,
allowing querying across boundaries.

Example (Snowflake External Tables): Snowflake
allows defining "external tables" that point
to data in locations like S3. Metadata
(location, format) is stored, but the data
isn't ingested. When queried, Snowflake
reads directly from the external source
(S3) at that moment.


\subsection*{Virtualization}
Data Virtualization is closely related
to federated queries but typically involves query
systems (like Trino, formerly PrestoSQL, or Presto)
that do not store data internally. They act purely
as a query layer over various external data sources.

Key Considerations:
\begin{itemize}
    \item Performance and the range of supported
    external data sources are the most significant
    factors when evaluating or using data
    virtualization.

    \item Ideal for organizations with data spread
    across multiple disparate sources, providing
    unified access.

    \item Caution is needed when querying production
    systems directly and frequently, as it can
    negatively impact their performance
    (the source is hit on each query).

    \item Can be used effectively as part of data
    ingestion/ETL pipelines (e.g., using Trino for
    a scheduled data pull from a production DB into
    data lake storage like S3).
\end{itemize}

While Federation and Virtualization are often used
interchangeably, virtualization sometimes implies
a broader capability, potentially including a
unified semantic layer, data abstraction,
and more complex integrations than simple query
federation (which can be seen as a component technology).










% \subsection*{Streaming Transformations and Processing}









 
\subsection*{Undercurrents}








