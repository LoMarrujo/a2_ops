\section{End-to-End Pipeline}
\noindent
This first iteration will be lackluster by design. Its goal is to
allow us to have all the pieces of a pipeline in place so that we
can prioritize which ones to improve next. Having a full prototype
is the easiest way to identify the impact bottleneck.


\subsection{The Simplest Scaffolding}
For the first prototype of an application, we will focus on being
able to deliver results to users. This means that we will start
with the inference pipeline.

Building, validating, and updating hypotheses about the best way to
model data are core parts of the iterative model building process,
which starts before we even build our first model!

The vast majority of ML projects should start with a heuristic
based on expert knowledge and data exploration and to use it
to confirm initial assumptions and speed up iteration.

Once you have a heuristic, it is time to create a pipeline that can
gather input, preprocess it, apply your rules to it, and serve
results.

The point here is to do for your product the same thing we did for
your ML approach, simplify it as much as possible, and build it so
you have a simple functional version. This is often referred to as
an MVP (minimum viable product) and is a battle-tested method for
getting useful results as fast as possible.



\subsection{Protoype}

\begin{enumerate}
    \item Parse and Clean Data
    \item Tokenizing Text
    \item Generating Features
\end{enumerate}



\subsection{Test Your Workflow}
We can test our assumptions about the way we've framed our problem
and how useful our proposed solution is. Take a look both at the
objective quality of our initial rules and examine whether we
are presenting our output in a useful manner.


\subsection*{User Experience}
Let's first examine how satisfying our product is to use,
independently of the quality of our model. In other words, if we
imagine that we will eventually get a model that performs well
enough, is this the most useful way to present results to our users?

In other words, we would want to make sure that the results we
present are useful (or will be if we improve our model). On the
flip side, of course, we'd also like our model to perform well.
That is the next aspect we'll evaluate.



\subsection*{Modeling Results}
Having a working prototype early on will allow us to identify and
iterate on our chosen metrics to make sure they represent product
success.

The goal of considering both user experience and model performance
is to make sure we are working on the most impactful aspect. If
your user experience is poor, improving your model is not helpful.
In fact, you may realize you would be better served with an entirely
different model!

The goal of looking both at modeling results and at the current
presentation of the product is to identify which challenge to
tackle next. Most of the time, this will mean iterating on the
way we present results to our users (which could mean changing the
way we train our models) or improving model performance by
identifying key failure points.

