\section{Create a Plan}
Many ML projects are doomed from the start due to a misalignment between product
metrics and model metrics. More projects fail by producing good models that
aren't helpful for a product rather than due to modeling difficulties.

\

\subsection{Measuring Success}
We have three potential approaches for increasing the complexity of ML models
\footnote{
    It is important to also realize that even features that could benefit from
    ML can often simply use a heuristic for their first version. Once the
    heuristic is being used, you may even realize that you do not need ML at all.
}:
\begin{enumerate}
    \item[1.] Baseline, design heuristics based on domain knowledge.
    \item[2.] Simple Model. 
    \item[3.] Complex Model.
\end{enumerate} 
All of these approaches are different and may evolve as we learn more from
prototypes along the way, but when working on ML, you should define a common
set of metrics to compare the success of modeling pipelines.

To that end, we will cover four categories of performance that have a
large impact on the usefulness of any ML product: business metrics,
model metrics, freshness, and speed. Clearly defining these metrics will
allow us to accurately measure the performance of each iteration.

\begin{itemize}
    \item \textbf{Business Performance}
    
    \noindent
    Once the product or feature goal is clear, a metric should be defined
    to judge its success. This metric should be separate from any model
    metrics and only be a reflection of the product's success
    \footnote{
        Product metrics may be as simple as the number of users a feature
        attracts or more nuanced such as the click-through rate (CTR) of
        the recommendations we provide.
    }.

    Product metrics are ultimately the only ones that matter, as they
    represent the goals of your product or feature. All other metrics
    should be used as tools to improve product metrics
    \footnote{
        Product metrics, however, do not need to be unique. While most
        projects tend to focus on improving one product metric, their
        impact is often measured in terms of multiple metrics,
        including guardrail metrics, metrics that shouldn't decline
        below a given point.
    }. 


    \item \textbf{Model Performance}
    
    \noindent
    The ultimate product metric that determines the success of a model
    is the proportion of visitors who use the output of a model out of
    all the visitors who could benefit from it.

    When a product is still being built and not deployed yet, it is not
    possible to measure usage metrics. To still measure progress, it is
    important to define a separate success metric called an offline
    metric or a model metric.

    Small changes to the interaction between the model and product can
    make it possible to use a more straightforward modeling approach
    and deliver results more reliably
    \footnote{e.g. Changing an interface so that a model's results can
    be omitted if they are below aconfidence threshold; Presenting a
    few other predictions or heuristics in addition to a model's top
    prediction; Communicating to users that a model is still in an
    experimental phase and giving them opportunities to provide feedback
    }.


    \item \textbf{Freshness and Distribution Shift}
    
    \noindent
    Even if a model is trained on an adequate dataset, many problems have
    a distribution of data that changes as time goes on. When the distribution
    of the data shifts, the model often needs to change as well in order to
    maintain the same level of performance. In general, a model can perform
    well on data it hasn't seen before as long as it is similar enough to
    the data it was exposed to during training.

    Not all problems have the same freshness requirements.

    Depending on your business problem, you should consider how hard it will
    be to keep models fresh. How often will you need to retrain models, and
    how much will it cost you each time we do so?


    \item \textbf{Speed}
    
    \noindent
    Ideally, a model should deliver a prediction quickly. This allows users
    to interact with it more easily and makes it easier to serve a model to
    many concurrent users. So how fast does a model need to be?


\end{itemize}



\subsection{Estimate Scope and Challenges}
In ML, success generally requires understanding the context of the task well,
acquiring a good dataset, and building an appropriate model. Let's elaborate.

\begin{itemize}
    \item \textbf{Leverage Domain Expertise}
    
    \noindent
    The simplest model we can start with is a heuristic: a good rule of thumb based on
    knowledge of the problem and the data. The best way to devise heuristics is to see
    what experts are currently doing. Most practical applications are not entirely novel.
    How do people currently solve the problem you are trying to solve?

    The second best way to devise heuristics is to look at your data. Based on your dataset,
    how would you solve this task if you were doing it manually?

    To identify good heuristics, I recommend either learning from experts in the field or
    getting familiar with the data.


    \item \textbf{Stand on the Shoulders of Giants}
    
    \noindent

    Have people solved similar problems? If so, the best way to get started is to understand
    and reproduce existing results. Look for public implementations either with
    similar models or similar datasets, or both.

\end{itemize}
Leveraging existing open code and datasets can help make
implementation faster. In the worst case, if none of the existing models performs well
on an open dataset, you now at least know that this project will require significant
modeling and/or data collection work.

If you have found an existing model that solves a similar task and managed to train it
on the dataset it was originally trained on, all that is left is to adapt it to your domain.
To do so, I recommend going through the following successive steps:
\begin{enumerate}
    \item[1.] Find a similar open source model, ideally paired with a dataset it was trained on,
    and attempt to reproduce the training results yourself.

    \item[2.] Once you have reproduced the results, find a dataset that is closer to your use
    case, and attempt to train the previous model on that dataset.

    \item[3.] Once you have integrated the dataset to the training code, it is time to judge how
    your model is doing using the metrics you defined and start iterating.
\end{enumerate}



\subsection{To Make Regular Progress: Start Simple}
It is worth repeating that much of the challenge in ML is similar to one
of the biggest challenges in software—resisting the urge to build pieces
that are not needed yet.

Many ML projects fail because they rely on an initial data acquisition and
model building plan and do not regularly evaluate and update this plan.

Because of the stochastic nature of ML, it is extremely hard to predict how
far a given dataset or model will get us.

For that reason, it is vital to start with the simplest model that could
address your requirements, build an end-to-end prototype including this
model, and judge its performance not simply in terms of optimization metrics
but in terms of your product goal.

\subsection*{Start with a Simple Pipeline}
To do this, we will need to build a pipeline that can take data in and
return results. For most ML problems, there are actually two separate
pipelines to consider:
\begin{enumerate}
    \item[i. Training] 
    
    A training pipeline ingests all of the labeled data you would like
    to train on and passes it to a model. It then trains said model on
    the dataset until it reaches satisfactory performance.
    
    Most often, a training pipeline is used to train multiple models
    and compare their performance on a held-out validation set.


    \item[ii. Inference/Prediction] 

    At a high level, an inference pipeline starts by accepting input
    data and preprocessing it.
    
    The preprocessing phase usually consists of multiple steps. Most
    commonly, these steps will include cleaning and validating the
    input, generating features a model needs, and formatting the data
    to a numerical representation appropriate for an ML model.
    
    Pipelines in more complex systems also often need to fetch
    additional information the model needs such as user features
    stored in a database.


    \item[iii. Evaluation]
    
    From the predictions, what can we improve?
\end{enumerate}