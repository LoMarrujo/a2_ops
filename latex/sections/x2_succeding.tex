\section{How to use AI in your business}



\subsection{What do you need to know about AI?}
What you do need to know to manage AI projects is the same as with
any other project: how to define metrics and processes that allow
you to properly comprehend and monitor the direction and success
of the project. Once you understand that, managing AI projects is
similar to running those projects you've overseen before.

Most AI concepts that are relevant to making executive decisions
could be explained to businesspeople in business terms. Ideally,
your data scientists should be able to do that. If they can't,
you should supplement your project team with people who have
expertise in both AI and business to help with communications.

What you need to know to manage an AI project is how to relate
AI concepts to business. Namely, you need to be able to answer
the following questions:
\begin{itemize}
    \item What can AI do, and how can I use that in my business?
    \item What type of AI project should I start with first?
    \item How will I measure how successful AI is in helping my business?
    \item How should I manage an AI project?
    \item What resources are scarce, and how should I best assign them?
\end{itemize}


\subsection{How is AI used?}
You make money when you perform an appropriate business action.
That leads us to where AI plays into any system it's using—AI
directs you in which action to take.

While AI, ML, and data science are perceived as new, the role
that they play in making businesses successful isn't new.

The Sense/Analyze/React loop. Any successful analytical project
must have all three elements of this loop.

\begin{itemize}
    \item Sense - The sensor part of the loop is where you get
    the data that the analysis looks for.

    \item Analyze - That's the box in which you now apply AI
    to your dataset.

    \item React - Reactor/effector is the part responsible for
    the action in the real world. That reaction might be
    performed by a human or by a machine.
\end{itemize}

The advance of AI broadened the applicability of the
Sense/Analyze/React loop, because AI brought to the table new
analytical capabilities.

What's new with AI and big data is that automated analysis has
become cheaper, faster, better, and (using big data systems)
capable of operating on much larger datasets.


\subsection{The Innovation of AI}
The advance of AI has bradened the applicability of the SAR loop,
because AI brought new capabilities of cheaper, faster, better and
scalable automated analysis
    \footnote{
        e.g. language models and translations, computer vision and 
        autonomous cars, product recommendations, etc.
    }.

What's not new or different with AI is that analysis still
can't make money all by itself.

Sometimes you won't be able to make a profit regardless of how
good your AI-powered analysis is. When you perform an analysis,
you incur the cost of that analysis. Profit may happen when you
react based on the results of analysis. If there's no business
action you can take after getting the results of analysis, such
an analysis is always a loss.


\subsection{Making money with AI}
