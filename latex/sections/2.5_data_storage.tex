\section{Storage}
Storage is the cornerstone of the data engineering lifecycle and underlies
its major stages—ingestion, transformation, and serving.



\subsection{Raw Ingredients of Data Storage}

\begin{enumerate}
    \item \textbf{Storage Media and Their Trade-offs}
    \begin{itemize}
        \item \textbf{Magnetic Disk Drives (HDDs)}: Low-cost, high-capacity storage commonly used in cloud object storage, but slow for random access and transactional workloads.
        \item \textbf{Solid-State Drives (SSDs)}: High-speed storage with low latency, essential for OLTP databases and caching, but more expensive per gigabyte.
        \item \textbf{Random Access Memory (RAM)}: Ultra-fast but volatile storage used for caching and in-memory processing, significantly more expensive than SSDs.
    \end{itemize}
    
    \item \textbf{Distributed Storage and Compute}
    \begin{itemize}
        \item Cloud object storage enables parallel reads across many disks, improving data access speeds.
        \item Network bandwidth and latency become critical performance bottlenecks in distributed storage systems.
    \end{itemize}
    
    \item \textbf{Serialization and Data Formats}
    \begin{itemize}
        \item \textbf{Serialization} optimizes data storage and transfer by converting data into a storable format.
        \item \textbf{Row-based formats} (CSV, JSON) are useful for transactional queries but inefficient for large-scale analytics.
        \item \textbf{Columnar formats} (Parquet) provide better compression and query efficiency for analytical workloads.
    \end{itemize}
    
    \item \textbf{Compression and Its Impact}
    \begin{itemize}
        \item \textbf{Storage efficiency}: Compression reduces storage costs and speeds up data scans by reducing data size.
        \item \textbf{Compute trade-offs}: Compressing and decompressing data adds CPU overhead, requiring careful tuning.
    \end{itemize}
    
    \item \textbf{Caching and Storage Hierarchy}
    \begin{itemize}
        \item \textbf{Fast caches} (CPU cache, RAM) store frequently accessed data to reduce latency.
        \item \textbf{Cloud storage tiers} balance cost and retrieval speed, optimizing data access strategies.
        \item \textbf{Archival storage} is cost-effective for backups and compliance but has high retrieval latency.
    \end{itemize}
    
    \item \textbf{Networking and CPU Considerations}
    \begin{itemize}
        \item \textbf{Distributed storage performance} is influenced by network bandwidth and CPU efficiency.
        \item \textbf{Cloud availability zones} enhance durability and ensure data availability in case of failures.
    \end{itemize}
    
    \item \textbf{Why This Matters}
    \begin{itemize}
        \item Choosing the right storage medium impacts cost, performance, and scalability.
        \item Optimizing serialization, compression, and caching improves query efficiency and system throughput.
        \item Distributed architectures require balancing storage, networking, and compute resources for efficiency.
    \end{itemize}
\end{enumerate}

A data scientist/engineer should understand these storage fundamentals to build scalable and cost-effective data systems.




\subsection{Data Storage Systems}

\begin{enumerate}
    \item \textbf{Single Machine vs. Distributed Storage}

    \noindent
    As data storage needs grow beyond the capacity of a single server,
    distributed storagesolutions become essential
    \footnote{
        Distributed storage systems are integral to modern architectures,
        supporting technologies such as object storage, Apache Spark,
        and cloud data warehouses.
    }.
    In distributed
    storage, multiple servers coordinate to store, retrieve, and
    process data efficiently while providing redundancy in case of
    failures.
    

    \item \textbf{Eventual vs. Strong Consistency}

    \noindent
    A fundamental challenge in distributed systems is maintaining data
    consistency across multiple nodes. This leads to a trade-off between
    consistency and availability. Two key consistency paradigms exist:
    \begin{itemize}
        \item \textbf{Eventual Consistency:} Ensures that all replicas
        of a dataset converge to the same value over time. It is common
        in high-scale distributed systems where immediate consistency is
        not required.

        \item \textbf{Strong Consistency:} Guarantees that reads always
        return the latest write. It is necessary for applications that
        require precise data accuracy but comes with increased latency.
    \end{itemize}
    Understanding how databases handle consistency is critical for data
    engineers, who must balance consistency, performance, and availability
    based on system requirements.


    \item \textbf{File Storage}

    \noindent
    File storage organizes data into a hierarchical directory structure.
    It supports:
    \begin{itemize}
        \item \textbf{Finite length:} Each file has a defined size.
        \item \textbf{Append operations:} New data can be added at the end
        of a file.
        \item \textbf{Random access:} Data can be read or modified at any
        position within a file.
    \end{itemize}


    \item \textbf{Local Disk Storage}

    \noindent
    Local filesystems, such as NTFS (Windows) and ext4 (Linux), store data
    on SSDs or magnetic disks. These systems provide:
    \begin{itemize}
        \item Full read-after-write consistency.
        \item Locking mechanisms for concurrent access.
        \item Advanced features like journaling, snapshots, redundancy,
        encryption, and compression.
    \end{itemize}


    \item \textbf{Network-Attached Storage (NAS)}

    \noindent
    NAS provides shared file storage over a network, offering advantages
    such as redundancy, resource management, and access control. However,
    performance may be impacted by network latency. Engineers should understand
    the consistency models of their NAS solutions, particularly when multiple
    clients access shared data.


    \item \textbf{Cloud Filesystem Services}

    \noindent
    Cloud-based filesystems, such as Amazon EFS, function similarly to NAS but
    are fully managed by cloud providers. These services provide:
    \begin{itemize}
        \item Automatic scaling.
        \item Pay-per-use pricing.
        \item Strong consistency guarantees within a single instance.
    \end{itemize}


    \item \textbf{Block Storage}

    \noindent
    Block storage operates at a lower level than file storage, interacting directly
    with storage hardware. Each block is a fixed-size unit of data, typically used
    by transactional databases and operating systems.


    \item \textbf{RAID (Redundant Array of Independent Disks)}

    \noindent
    RAID systems combine multiple disks to enhance performance, increase capacity,
    and improve fault tolerance. Various RAID configurations balance speed and redundancy differently.


    \item \textbf{RAID (Redundant Array of Independent Disks)}

    \noindent
    SAN systems provide block-level storage over a network, pooling resources for scalability, redundancy,
    and optimized performance. These are commonly used in enterprise data centers.


    \item \textbf{Cloud Virtualized Block Storage}
    
    \noindent
    Cloud services like Amazon EBS offer virtualized block storage for virtual machines, supporting:
    \begin{itemize}
        \item High durability through data replication.
        \item Performance tiers optimized for different workloads.
        \item Snapshot capabilities for backup and disaster recovery.
    \end{itemize}

    \item \textbf{Local Instance Volumes}
    
    \noindent
    Cloud providers offer ephemeral local storage directly attached to VMs. While these volumes
    provide high performance, they do not persist after VM termination and lack advanced redundancy features.

    \item \textbf{Object Storage}
    
    \noindent
    Object storage is a modern alternative to traditional file storage, designed for scalability and resilience.
    Cloud-based object storage services, such as Amazon S3, Azure Blob Storage, and Google Cloud Storage,
    play a critical role in data engineering applications.

    Key characteristics of object storage:
    \begin{itemize}
        \item \textbf{Immutable objects:} Objects are written once and cannot be modified; updates require rewriting the entire object.
        \item \textbf{Scalability:} Supports exabytes of data with automatic scaling.
        \item \textbf{Parallel access:} Optimized for batch reads and writes, making it ideal for big data applications.
    \end{itemize}
    Object storage enables separation of compute and storage, allowing for on-demand scalability in cloud data processing.
    It is widely used for data lakes, backups, and serving large-scale web content.


    \item \textbf{Cache and Memory-Based Storage Systems}
    
    \noindent
    RAM-based storage systems provide ultra-fast retrieval speeds but are volatile and vulnerable to data loss.
    They are primarily used for caching, rather than long-term storage
    \footnote{
        Example: 
        Redis is a more advanced key-value store supporting lists, sets, and other data structures.
        It includes built-in persistence mechanisms like snapshotting and journaling, making it
        suitable for applications that require a balance between speed and durability.
    }.



    \item \textbf{The Hadoop Distributed File System (HDFS)}
    
    \noindent
    HDFS was a cornerstone of early big data systems, designed to process large datasets
    efficiently using the MapReduce model. Unlike object stores, HDFS tightly integrates storage
    with compute, distributing data across a cluster while maintaining replication for fault tolerance.

    HDFS remains relevant, especially in legacy systems and hybrid cloud architectures.
    While pure Hadoop ecosystems have declined, HDFS is still used in platforms like Amazon EMR and Apache Spark.



    \item \textbf{Streaming Storage}
    
    \noindent
    Streaming storage systems, such as Apache Kafka, Amazon Kinesis, and Google Pub/Sub, enable scalable
    and durable message retention. Kafka, for instance, supports long-term storage by offloading older
    messages to object storage while maintaining replay capabilities.



    \item \textbf{Indexes}
    
    \noindent
    Indexes provide a map of the table for particular fields and allow extremely fast
    lookup of individual records. Without indexes, a database would need to scan an
    entire table to find the records satisfying a WHERE condition.

    Traditional relational databases rely on indexes to speed up lookups.
    However, columnar databases optimize for analytical workloads by storing data in
    column-major format. This minimizes disk reads for queries that access only
    specific columns. Columnar databases are optimized for:
    \begin{itemize}
        \item Large-scale aggregations and transformations.
    
        \item High-compression ratios due to columnar similarity.
        
        \item Efficient pruning of unnecessary data during queries.
    \end{itemize}


    \item \textbf{Partitioning and Clustering}
    
    \noindent
    Partitioning divides tables into smaller subsets (e.g., by date),
    reducing query scan time. Clustering sorts data within partitions,
    colocating similar values for faster filtering and joining
    \footnote{
        Example: Snowflake Micro-Partitioning.
        Snowflake organizes tables into micro-partitions (50-500MB in size),
        clustering similar values automatically.

        Query performance improves by pruning irrelevant micro-partitions.

        Metadata storage aids efficient query planning, functioning similarly to traditional indexes.
    }.

\end{enumerate}







\subsection{Data Engineering Storage Abstractions}
Data engineering storage abstractions are patterns for organizing
and querying data, built upon underlying storage systems, primarily
for data science, analytics, and reporting.

Key abstractions discussed include data warehouses, data lakes,
data lakehouses, and data platforms.

Choosing the right abstraction depends on:
\begin{itemize}
    \item \textbf{Purpose/Use Case:} What the data will be used for.
    \item \textbf{Update Patterns:} Whether it needs optimization
    for bulk updates, streaming inserts, or upserts.
    \item \textbf{Cost:} Financial costs, time-to-value, and
    opportunity costs.
    \item \textbf{Separation of Storage and Compute:} A major trend
    blurring the lines between systems like data warehouses and data
    lakes, impacting speed, cost, and purpose.
\end{itemize}


\subsection*{The Data Warehouse}
A standard OLAP
\footnote{
    On-Line Analytical Processing
}
architecture for data centralization, and an organizational pattern
within a company.

Evolved from traditional databases to MPP systems and now cloud data
warehouses. Cloud versions can handle raw text/JSON, acting somewhat
like lakes, but struggle with truly unstructured data (images, video)
unless paired with object storage.

In practice, cloud data warehouses are often used to organize data
into a data lake, a storage area for massive amounts of unprocessed
raw data
\footnote{
    The limitation is that cloud data warehouses cannot handle truly
    unstructured data, such as images, video, or audio, unlike a true
    data lake.
}.


\subsection*{The Data Lake}
The data lake was originally conceived as a massive store where data
was retained in raw and unprocessed form, [often built on Hadoop]
for cost-effective storage.

Recent trends include moving storage to cloud object storage
(separating compute) and realizing the need for features like
schema management and updates, which spurred the lakehouse concept.


\subsection*{The Data Lakehouse}
Combines data lake and data warehouse features. As it is generally
conceived, the lakehouse stores data in object storage just like
a lake
\footnote{
    We would be remiss not to point out that the architecture of
    the data lakehouse is similar to the architecture used by
    various commercial data platforms, including BigQuery and
    Snowflake. These systems store data in object storage and
    provide automated metadata management, table history,
    and update/delete capabilities.
    
    The complexities of managing underlying files and storage are
    fully hidden from the user.
}.

Its key advantage over proprietary platforms is interoperability
due to open formats, allowing various tools to access data
directly. It allows imposing structure where needed while leaving
other data raw.



\subsection*{Data Platforms}
Vendor-driven ecosystems offering integrated tools around a core
storage layer, often including object storage integration.
They aim to simplify data engineering but can lead to vendor
lock-in. The concept is still evolving.

However, the race is on to create a walled garden of data tools,
both simplifying the work of data engineering and generating
significant vendor lock-in.



\subsection*{Stream-to-Batch Storage Architecture}
Similar to Lambda architecture, where data streams are sent to
multiple consumers. Some process data in real-time, while a batch
consumer writes data to long-term storage (like object storage or
a specialized buffer like BigQuery's) for batch querying. This
allows querying combined near real-time and historical data.











\subsection{Big Ideas and Trends in Storage}
The following text summarizes key concepts and trends influencing data
storage architecture in data engineering.



\subsection*{Data Catalog}
A data catalog is a metadata repository for all organizational data
(operational and analytical).
\begin{itemize}
    \item  It integrates with various systems, often using automated
    scanning and APIs, to collect and infer metadata
    (schemas, relationships, sensitive data).
    
    \item Catalogs provide a user interface for data discovery,
    understanding relationships, and collaboration, supporting both
    technical use cases (like table discovery in lakehouses) and
    organizational needs (like data search for analysts).
\end{itemize}


\subsection*{Data Sharing}
Data sharing allows organizations and individuals to share specific
data and carefully defined permissions with specific entities.

Facilitated by cloud environments, it aids collaboration but requires
strong security policies. It's a core feature of many cloud data
platforms.


\subsection*{Schema}
Describes the structure and organization of data
\footnote{
    Note that schema need not be relational.
    
    Rather, data becomes more useful when we
    have as much information about its
    structure and organization.
    
    For images stored in  a data lake, this schema
    information might explain the image format,
    resolution, and the way the images fit into
    a larger hierarchy.
}
(format, types, hierarchy, relationships).
What is the expected form of the data?

What is the file format? Is it structured, semistructured,
or unstructured?
What data types are expected? How does the data fit into a larger
hierarchy?
Is it connected to other data through shared keys or other
relationships?

Two main patterns exist:
\begin{enumerate}
    \item \textbf{Schema on Write:} (Traditional) Schema is enforced
    when data is written. Ensures data standards but is less flexible.

    \item \textbf{Schema on Read:} Schema is applied when data is read.
    Offers flexibility but requires careful handling, ideally with
    self-describing formats (like Parquet).
\end{enumerate}


\subsection*{Separation of Compute from Storage}
Separation of Compute from Storage is a dominant cloud pattern where
storage (often object storage) is decoupled from processing units.
\begin{itemize}
    \item \textbf{Motivation:} Enables ephemeral, scalable compute
    (pay-as-you-go efficiency for variable workloads) and leverages
    high durability/availability of cloud object stores.

    \item \textbf{Contrast:} Differs from traditional colocation
    where compute and data reside together for performance.

    \item \textbf{Hybrid Approaches:} Systems often use multi-tier
    caching (local storage for intermediate steps) or hybrid object
    storage (like BigQuery/Colossus, S3 Select) to blend benefits.
    Examples include EMR using temporary HDFS, Spark using
    memory/ephemeral filesystems, and Druid using SSDs backed by
    object storage for durability.
\end{itemize}


Snowflake's multi-cluster, shared data architecture is built on three
distinct layers:
\begin{enumerate}
    \item \textbf{Storage Layer (Data Storage):} This data resides in
    cloud storage services provided by the underlying cloud platform
    (like Amazon S3, Azure Blob Storage, or Google Cloud Storage),
    depending on where your Snowflake account is hosted.

    This layer is designed for durability, scalability, and
    cost-effectiveness, leveraging the native capabilities of the
    cloud provider's storage. You pay for storage based on the volume
    of compressed data you store.   

    \item \textbf{Compute Layer (Query Processing):} this layer
    consists of Virtual Warehouses. These are the engines that execute
    your SQL queries and perform data processing tasks.
    
    Each warehouse operates independently without competing for
    resources with other warehouses.
    You can create multiple Virtual Warehouses of different sizes
    (e.g., X-Small, Small, Medium, Large, etc.).

    You pay for compute based on the size of the warehouse and how
    long it runs (per-second billing after the first minute).
    Warehouses can be suspended (turned off) when not in use to save
    costs.

    \item \textbf{Cloud Services Layer (Coordination):} This layer acts
    as the "brain" of Snowflake. It coordinates activities across the
    system.

    It manages infrastructure, optimizes queries, handles security
    (authentication, access control), manages metadata
    (information about tables, schemas, etc.), and ensures transactional
    consistency.
\end{enumerate}



\subsection*{Data Storage Lifecycle and Data Retention}

\begin{enumerate}
    \item \textbf{Lifecycle (Temperature):} Classifies data based on
    access frequency:
    \begin{itemize}
        \item \textbf{Hot:} Frequent access, fastest/most expensive storage
        (SSD, memory).
        \item \textbf{Warm:} Semi-regular access, intermediate storage tiers.
        \item \textbf{Cold:} Infrequent access (archival), cheapest storage,
        slow/costly retrieval.
    \end{itemize}
    Use automated lifecycle policies to manage tiers and costs effectively.

    \item \textbf{Retention:} Deciding how long to keep data, considering: 
    \begin{itemize}
        \item \textbf{Value:} Importance and replaceability.
        \item \textbf{Time:} Value often decreases with age; TTL for hot
        storage.
        \item \textbf{Compliance:} Regulatory requirements for keeping or
        deleting data.
        \item \textbf{Cost:} Balancing storage costs against data value and
        retention needs. Avoid hoarding useless data.
    \end{itemize}

\end{enumerate}



\subsection*{Single-Tenant Versus Multitenant Storage}

\begin{itemize}
    \item \textbf{Single-Tenant:} Each tenant gets dedicated, isolated
    storage resources. Allows customization but can complicate cross-tenant
    operations.

    \item \textbf{Multitenant:} Multiple tenants share storage resources.
    More efficient infrastructure use but requires careful isolation and
    uniform structure.
\end{itemize}





\subsection{Undercurrents}
Because storage is ubiquitous, these undercurrents differ slightly
from those affecting data in motion or transformation.

\begin{itemize}
    \item \textbf{Security:}
    \begin{itemize}
        \item Viewed as an \textit{enabler} for wider data sharing
        by ensuring safety, thus increasing data value.
        \item Requires robust protection for data at rest and in
        motion.
        \item Emphasizes fine-grained access control (column, row,
        cell-level) and the principle of least privilege.
    \end{itemize}

    \item \textbf{Data Management:}
    \begin{itemize}
        \item Critical for reading from and writing to storage
        systems effectively.

        \item \textit{Metadata Management:} Enhances data value
        through catalogs (discovery), lineage (tracing),
        dictionaries (knowledge sharing), and supports governance.
        Active metadata analysis provides insights.

        \item \textit{Data Versioning:} Especially in object storage,
        aids error recovery and tracking dataset history
        (valuable for ML model debugging).
    \end{itemize}

    \item \textbf{Privacy:}
    \begin{itemize}
        \item Regulations (e.g., GDPR) significantly impact storage
        design and data handling.

        \item Necessitates managing the lifecycle of private data,
        including processes for deletion requests.

        \item Involves techniques like anonymization and masking to
        protect sensitive information.
    \end{itemize}

    \item \textbf{DataOps:}
    \begin{itemize}
        \item Focuses on operational aspects, including monitoring of
        storage systems and data quality.

        \item \textit{Systems Monitoring:} Covers infrastructure,
        serverless components (object storage), cost management
        (FinOps), security configurations, and access patterns.

        \item \textit{Data Observability/Monitoring:} Actively 
        understanding data characteristics, tracking statistics,
        detecting anomalies, and validating consistency.
    \end{itemize}

    \item \textbf{Data Architecture:}
    \begin{itemize}
        \item Recognizes storage as a fundamental architectural
        component.

        \item Requires designing for specific reliability and
        durability needs.

        \item Involves understanding upstream sources and
        downstream access patterns/query types.

        \item Includes planning for data growth, managing costs
        (FinOps), and preparing for scale.

        \item Generally favors fully managed systems, understanding
        their Service Level Agreements (SLAs).
    \end{itemize}

    \item \textbf{Orchestration:}
    \begin{itemize}
        \item Deeply interconnected with storage, as it manages
        data flow through pipelines involving storage steps.

        \item Helps manage the complexity of integrating multiple
        storage systems and query engines.
    \end{itemize}

    \item \textbf{Software Engineering:}
    \begin{itemize}
        \item Involves writing code that interacts correctly and
        efficiently with storage systems.

        \item Promotes using Infrastructure as Code (IaC) to define
        and manage storage resources.
        
        \item Leverages ephemeral compute resources alongside
        persistent storage (often object storage), enabled by
        compute-storage separation, for clean and scalable
        infrastructure.
    \end{itemize}
\end{itemize}


