\section{Data Engineering}


\subsection{What's Data Engineering?}
\noindent
% Fundamentals of DataEng
Data engineering is
\footnote{
    Data engineering can also be viewed as the intersection of
    security, data management, DataOps, data architecture
    orchestration, and software engineering.
}
the design, development, and maintenance of systems that
take in raw data and produce quality information that
supports downstream use cases, such as analysis and
machine learning.

A data engineer manages the data engineering lifecycle
\footnote{
    The data engineering lifecycle shifts toward
    the data itself and the end goals that it must serve.
    The stages of the data engineering lifecycle are as follows:
    \begin{itemize}
        \item Generation.
        \item Storage.
        \item Ingestion.
        \item Transformation.
        \item Serving.
    \end{itemize}
}, 
beginning with getting data from source systems and ending
with serving data for use cases, such as analysis or machine
learning.



\subsection*{Data Maturity and the Data Engineer}
The level of data engineering complexity within a company depends
a great deal on the company's data maturity. This significantly
impacts a data engineer's day-to-day job responsibilities and
career progression. What is data maturity, exactly?

\textbf{Data maturity} is the progression toward higher data
utilization, capabilities, and integration across the organization,
but data maturity does not simply depend on the age or revenue of
a company. An early-stage startup can have greater data maturity
than a 100-year-old company with annual revenues in the billions.
What matters is the way data is leveraged as a competitive
advantage.

Data maturity models have many versions, such as Data Management
Maturity (DMM) and others. A simple data maturity model has three
stages:
\begin{enumerate}
    \item \textbf{Starting with data.}
    
    \noindent
    A company getting started with data is, by definition, in the
    very early stages of  its data maturity. The company may have
    fuzzy, loosely defined goals or no goals.

    Data architecture and infrastructure are in the very early
    stages of planning and development. Adoption and utilization
    are likely low or nonexistent. The data team is small, often
    with a headcount in the single digits.

    A data engineer should focus on the following in organizations
    getting started with data
    \footnote{
        This is a delicate stage with lots of pitfalls.
        Here are some tips for this stage:
        \begin{itemize}
            \item Organizational willpower may wane if a lot of
            visible successes don't occur with data. Getting quick
            wins will establish the importance of data within the
            organization. Just keep in mind that quick wins will
            likely create technical debt. Have a plan to reduce
            this debt, as it will otherwise add friction for future
            delivery.
            \item Get out and talk to people, and avoid working in
            silos. We often see the data team working in a bubble,
            not communicating with people outside their departments
            and getting perspectives and feedback from business
            stakeholders.
            \item Avoid undifferentiated heavy lifting. Don't box
            yourself in with unnecessary technical complexity.
            Use off-the-shelf, turnkey solutions wherever possible.
            \item Build custom solutions and code only where this
            creates a competitive advantage.
        \end{itemize}
    }:
    \begin{itemize}
        \item Get buy-in from key stakeholders, including executive
        management.
        \item Define the right data architecture (usually solo,
        since a data architect likely isn't available). This means
        determining business goals and the competitive advantage
        you're aiming to achieve with your data initiative. Work
        toward a data architecture that supports these goals.
        \item Identify and audit data that will support key
        initiatives and operate within the data architecture you
        designed.
        \item Build a solid data foundation for future data analysts
        and data scientists to generate reports and models that
        provide competitive value.
    \end{itemize}

    
    \item \textbf{Scaling with data.}
    
    \noindent
    The company has moved away from ad hoc data requests
    and has formal data practices. Now the challenge is creating
    scalable data architectures and planning for a future where the
    company is genuinely data-driven. Data engineering roles move from
    generalists to specialists, with people focusing on particular
    aspects of the data engineering lifecycle.

    A data engineer's goals are
    \footnote{
        Issues to watch out for include the following:
        \begin{itemize}
            \item As we grow more sophisticated with data, there's a
            temptation to adopt bleeding-edge technologies based on
            trends. This is rarely a good use of your time and energy.
            Any technology decisions should be driven by the value
            they'll deliver to your customers.
            \item The main bottleneck for scaling is not cluster nodes,
            storage, or technology but the data engineering team.
            Focus on solutions that are simple to deploy and
            manage to expand your team's throughput.
            \item You'll be tempted to frame yourself as a technologist,
            a data genius who can deliver magical products. Shift your
            focus instead to pragmatic leadership and begin transitioning
            to the next maturity stage; communicate with other teams
            about the practical utility of data. Teach the organization
            how to consume and leverage data.
        \end{itemize}
    }:
    
    \begin{itemize}
        \item Establish formal data practices.
        \item Create scalable and robust data architectures.
        \item Adopt DevOps and DataOps practices.
        \item Build systems that support ML.
        \item Continue to avoid undifferentiated heavy lifting and
        customize only when a competitive advantage results.
    \end{itemize}


    \item \textbf{Leading with data.}
    
    \noindent
    At this stage, the company is data-driven. The automated pipelines
    and systems created  by data engineers allow people within the
    company to do self-service analytics and ML. Introducing new data
    sources is seamless, and tangible value is derived. Data engineers
    implement proper controls and practices to ensure that data is
    always available to the people and systems.

    A data engineer will continue building on prior stages, plus they
    will do the following
    \footnote{
        Issues to watch out for include the following:
        \begin{itemize}
            \item At this stage, complacency is a significant danger.
            \item Technology distractions are a more significant danger
            here than in the other stages. There's a temptation to
            pursue expensive hobby projects that don't deliver value
            to the business. Utilize custom-built technology only where
            it provides a competitive advantage.
        \end{itemize}
    }:
    \begin{itemize}
        \item Create automation for the seamless introduction and usage
        of new data.
        \item Focus on building custom tools and systems that leverage
        data as a competitive advantage.
        \item Focus on the “enterprisey” aspects of data, such as data
        management (including data governance and quality) and DataOps.
        \item Deploy tools that expose and disseminate data throughout
        the organization, including data catalogs, data lineage tools,
        and metadata management systems.
        \item Collaborate efficiently with software engineers,
        ML engineers, analysts, and others.
        (shouldn't this be from stage 1?).
        \item Create a community and environment where people can
        collaborate and speak openly, no matter their role or position
        (shouldn't this be from stage 1?).
    \end{itemize}
\end{enumerate}

