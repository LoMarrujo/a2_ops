\section{MLOps}



\subsection{What Is MLOps?}
Why isn't machine learning 10X faster?
Most of the problem-building machine learning systems involve
everything surrounding machine learning modeling: data engineering,
data processing, problem feasibility, and business alignment.

The quicker the feedback loop (Kaizen)
\footnote{
    What is Kaizen? In Japanese, it means improvement.
    Using Kaizen as a software management philosophy originated in
    the Japanese automobile industry after World War II. It
    underlies many other techniques: Kanban, root cause analysis,
    five why's, and Six Sigma.
    
    To practice Kaizen, an accurate and realistic assessment of
    the world's state is necessary and pursues daily, incremental
    improvements in the pursuit of excellence.
}
the more time to focus on business problems.

The reason models are not moving into production is the impetus
for the emergence of MLOps as a critical industry standard.
MLOps shares a lineage with DevOps in that DevOps philosophically
demands automation. A common expression is if it is not automated,
it's broken. Similarly, with MLOps, there must not be components of
the system that have humans as the levers of the machine
\footnote{
    The history of automation shows that humans are the least
    valuable doing repetitive tasks but are the most valuable
    using technology as the architects and practitioners.
}.



\subsection{An MLOps Hierarchy of Needs}
An ML system is a software system, and software systems work
efficiently and reliably when DevOps and data engineering best
practices are in place.

So how could it be possible to deliver the true potential of
machine learning to an organization if DevOps' basic foundational
rules don't exist or data engineering is not fully automated?


The ML hierarchy of needs in the next list is not a definitive guide
but is an excellent place to start a discussion.
\begin{enumerate}
    \item One of the major things holding back machine learning
    projects is this necessary foundation of \textbf{DevOps}.

    The foundation of DevOps is continuous integration. Without
    automated testing, there is no way to move forward with DevOps.
    Continuous integration is relatively painless for a Python
    project with the modern tools available. The first step is to
    build a "scaffolding" for a Python project which consists of:
    a \textbf{MakeFile}, \textbf{Requirements(.txt)},
    \textbf{code.py},  \textbf{test\_code.py}, \textbf{environment}
    \footnote{
        we recommend using conda for managing the environment and
        generating the requirements file.
    }.

    One of the most straightforward ways to implement CI for this
    Python scaffolding project is with \textbf{GitHub Actions}.

    CD, i.e., automatically pushing the machine learning project
    into production, would be the next logical step. This step
    would involve deploying the code to a specific location using
    a continuous delivery process and IaC (Infrastructure as Code).


    \item After the DevOps foundation is complete, next is
    \textbf{data automation}, that is, to automate the flow of data
    \footnote{
        e.g. with Apache Airflow, AWS Data Pipeline, AWS Guidelines,
        SchemaChange.
    }.

    Some items to consider here are the data's size, the frequency
    at which the information is changed, and how clean the data is.


    \item After DevOps and automated data comes
    \textbf{platform automation}, that is to evaluate is how an
    organization can use high-level platforms to build ML solutions.
    
    An ML platform solves real-world repeatability, scale, and
    operationalization problems.


    \item Finally, sssuming all of these other layers are complete
    (DevOps, Data Automation, and Platform Automation),
    \textbf{ML automation}, or \textbf{MLOps}, is possible.

    One way to articulate these best practices is to consider that
    they create reproducible models with robust model packaging,
    validation, and deployment. In addition, these enhance the
    ability to explain and observe model performance.

    The MLOps feedback loop includes the following:
    \begin{itemize}
        \item \textbf{Create and retrain models with reusable ML
        Pipeline.}
        
        \noindent
        Creating a model just once isn't enough. The data can
        change, the customers can change, and the people making
        the models can change. The solution is to have reusable
        ML pipelines that are versioned.
        

        \item \textbf{Continuous Delivery of ML Models.}
        
        \noindent
        CD of ML Models is similar to CD of software. When all of
        the steps are automated, including the infrastructure,
        using IaC, the model is deployable at any time to a new
        environment, including production.


        \item \textbf{Audit trail for MLOps pipeline.}
        
        \noindent
        It is critical to have auditing for machine learning models.
        There is no shortage of problems in machine learning,
        including security, bias, and accuracy. Therefore,
        having a helpful audit trail is invaluable, just as having
        adequate logging is critical in production software engineering
        projects. In addition, the audit trail is part of the
        feedback loop where you continuously improve your approach
        to the problem and the actual problem.
        

        \item \textbf{Monitor the model to improve future models.}
        
        \noindent
        One of the unique aspects of machine learning is that the
        data can literally "shift" beneath the model.

    \end{itemize}



\end{enumerate}
The culmination of MLOps is a machine learning system that works.
The people that work on operationalizing and building machine
learning applications are machine learning engineers and/or data
engineers.



