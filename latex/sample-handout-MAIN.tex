\documentclass[notoc]{tufte-handout}


% -------------------
% usepackage
% -------------------
%\geometry{showframe}% for debugging purposes -- displays the margins
\usepackage{amsmath}
\usepackage{graphicx}% Set up the images/graphics package
\setkeys{Gin}{width=\linewidth,totalheight=\textheight,keepaspectratio}
\graphicspath{{graphics/}}
\usepackage{booktabs} % The following package makes prettier tables.  We're all about the bling!
\usepackage{units} % The units package provides nice, non-stacked fractions and better spacing for units.
\usepackage{fancyvrb} % The fancyvrb package lets us customize the formatting of verbatim environments. We use a slightly smaller font.
\fvset{fontsize=\normalsize}
\usepackage{multicol} % Small sections of multiple columns
\usepackage{lipsum} % Provides paragraphs of dummy text


% -------------------
% Commands
% -------------------

% These commands are used to pretty-print LaTeX commands
\newcommand{\doccmd}[1]{\texttt{\textbackslash#1}}% command name -- adds backslash automatically
\newcommand{\docopt}[1]{\ensuremath{\langle}\textrm{\textit{#1}}\ensuremath{\rangle}}% optional command argument
\newcommand{\docarg}[1]{\textrm{\textit{#1}}}% (required) command argument
\newenvironment{docspec}{\begin{quote}\noindent}{\end{quote}}% command specification environment
\newcommand{\docenv}[1]{\textsf{#1}}% environment name
\newcommand{\docpkg}[1]{\texttt{#1}}% package name
\newcommand{\doccls}[1]{\texttt{#1}}% document class name
\newcommand{\docclsopt}[1]{\texttt{#1}}% document class option name

% Special Characters
\newcommand{\N}{\mathbb{N}}
\newcommand{\Z}{\mathbb{Z}}
\newcommand{\Q}{\mathbb{Q}}
\newcommand{\R}{\mathbb{R}}
\newcommand{\F}{\mathbb{F}}
\newcommand{\C}{\mathbb{C}}
\newcommand{\E}{\mathbb{E}}
\newcommand{\I}{\mathbb{I}}
\newcommand{\V}{\mathbb{V}}
\newcommand{\bs}{\boldsymbol}
\newcommand{\bb}{\boldsymbol}
\newcommand{\Pp}{\mathbb{P}}
\renewcommand{\O}{\mathcal{O}}
\newcommand{\fp}{\mathfrak{p}}

% Math Operators
\DeclareMathOperator{\Aut}{Aut}
\DeclareMathOperator{\Gal}{Gal}

% Special Commands
\newcommand{\xqed}{\hfill \lower-0.3ex\hbox{$\triangleleft$}}
\newcommand{\ov}[1]{\overline{#1}}
\newcommand*{\defeq}{\stackrel{\text{def}}{=}}
\newcommand{\pf}{\noindent\emph{Proof. }}
\newcommand{\pfsk}{\noindent\emph{Proof (Sketch). }}
\newcommand{\two}[4]{\begin{pmatrix} #1 & #2 \\ #3 & #4 \end{pmatrix}}

% Counter Commands
\newcounter{problem}
\newcommand{\prob}{
	\stepcounter{problem}%
	\noindent\textbf{Problem \theproblem: }
}

% -------------------
% Theorem Environments
% -------------------
\numberwithin{equation}{section}

\newtheorem{thm}{Teorema}[section]
\newtheorem{prop}{Proposición}[section]
\newtheorem{lem}{Lema}[section]
\newtheorem{cor}{Corolario}[section]

\newtheorem{examplex}{Ejemplo}[section]

\newenvironment{ex}
  {\pushQED{\qed}\renewcommand{\qedsymbol}{$\triangle$}\examplex}
  {\popQED\endexamplex}

\newtheorem{nex}{Non-Example}[section]
\newtheorem{dfnx}{Definición}[section]
\newenvironment{dfn}
{\pushQED{\qed}\renewcommand{\qedsymbol}{$\bullet}\dfnx}
{\popQED\enddfnx}



% -------------------
% Document
% -------------------

% Turn on numbering for section and subsection headings
\setcounter{secnumdepth}{2}
\title{Notes}
\author{ll}
\date{October 20024}  % if the \date{} command is left out, the current date will be used


% \setcounter{tocdepth}{2}
% \setcounter{secnumdepth}{2}




















\begin{document}

\maketitle% this prints the handout title, author, and date


\begin{abstract}
\noindent This document describes ...
\end{abstract}

\tableofcontents






% \tableofcontents

\newpage

% Ameisen, 2020
\section{Finding the ML Approach}
The ability to take a problem, estimate how best to solve it, 
build a plan to tackle it with ML, and confidently execute on said
plan. This is a skill that has to be learned through experience,
after multiple overly ambitious projects and missed deadlines.

The ML algorithm is only a small part of a ML system, other parts to 
consider are:
\begin{itemize}
    \item The Product Goal or Business requirements
    \footnote{
        regarding the stakeholders' requirements, which are a must or a nice-to-have?
    }.
    \item The interface.
    \item The data stack.
    \item The development, monitoring and maintaining models.
    \item The infrastructure.
\end{itemize}





% Building ML Powered Apps. Ameisen, 2020
\subsection{From Product Goal to ML Framing}
ML allows machines to learn from data and behave in a probabilistic
way to solve problems by optimizing for a given objective. This
stands in opposition to traditional programming, in which
step-by-step instructions solve a problem. This makes ML
particularly useful to build systems for which we are unable to
define a heuristic solution.

ML is powerful and can unlock entirely new products and services,
but since it is based on pattern recognition, it introduces a
level of uncertainty. It is important to identify which parts
of a product would benefit from ML and how to frame a learning
goal in a way that minimizes the risks of users having a poor
experience.

% Designing MLS. Huyen, 2022.
ML can be solve a lot of problems, but not all. There are two
necessary questions to ask when considering to use ML:
\begin{itemize}
    \item Is ML indispensable for a solution?
    \item Is it cost effective?
\end{itemize}
ML solutions generally learn complex patterns from data to predict
an unknown value
\footnote{
    Consider that insufficiently trained models come with bigger
    risks. Some ML companies fake it until they make it by using
    other tools.
}.
ML solutions shine if the problem:
\begin{itemize}
    \item is repetitive,
    \item the cost of wrong predictions is cheap
    \footnote{
        ML might still be suitable if, on average, the benefits 
        of correct predictions outweighs the cost of a wrong
        prediction.
    },
    \item is at scale,
    \item is somewhat stable,
    \item can be divided and conquered with ML and other tools
    \footnote{
        e.g. a set of rules that can be programmed.
    }.
\end{itemize}
ML should never be used if it is unethical to do so, or there are
other simpler solutions, or if it is not and won't be cost effective.

Consider the following examples where we identify the problem and
a potential solution.

% \begin{example}
It is close to impossible for humans to write step-by-step
instructions to automatically detect which animal is in an image
based on pixel values. By feeding thousands of images of different
animals to a convolutional neural network (CNN), however, we can
build a model that performs this classification more accurately
than a human. This makes it an attractive task to tackle with ML.
% \end{example}

% \begin{example}
Consider an application that calculates your taxes automatically
should rely on guidelines provided by the government. As you may
have heard, having errors on your tax return is generally frowned
upon. This makes the use of ML for automatically generating tax
returns a dubious proposition.
% \end{example}

In summary, never use ML when you can solve your problem with a
manageable set of deterministic rules, i.e., a set of rules that
could be confidently written and that would not be too complex to
maintain.



% Designing MLS. Huyen, 2022.
\subsection*{Product Goal Framing}
The first thing to do is to understand why the system is needed.
For businesses it must be driven by business objectives which
translate into ML objectives that guide the development. Also, 
the business objectives must be framed into a ML tasks
\footnote{
    Most businesses don't care about ML metrics, 
    unless they impact business metrics. This discrepancy can kill
    ML projects. The solution is to ask a lot of questions to
    understand the business side and how it related to ML. Most 
    businesses focus on maximizing profit or minimizing risks. So,
    a ML project must support this directly or indirectly.

    Remember that new ML metrics can always be defined and computed.
    Be realistic about expected improvements.
}.
This reduces the difficulty of the problem.

% https://developers.google.com/machine-learning/problem-framing/problem-framing
\textbf{Problem framing} is the process of analyzing a problem to isolate
the individual elements that need to be addressed to solve it
\footnote{
    Formal problem framing is the critical beginning for solving
    an ML problem, as it forces us to better understand both the
    problem and the data in order to design and build a bridge
    between them. -TensorFlow engineer
}.
Problem framing helps determine your project's technical
feasibility and provides a clear set of goals and success
criteria. When considering an ML solution, effective problem
framing can determine whether or not your product ultimately
succeeds.

% Designing MLS. Huyen, 2022.
The process to design ML systems that meet the business requirements
is iterative with continuous improvement and delivery
\footnote{
    Sometimes, a data scientist or ML engineer might spend less 
    time in model development and more in setting up the 
    infrastructure and related tasks.
}.
It is useful to identify the objectives, stakeholders and constraints.
Make sure there is an allignment of stakeholders and objectives.
Also, for a ML problem we need to precisely define the inputs,
outputs, the business objective that guides the development
and their relation to the ML tasks
\footnote{
    ML problems are tricky when there is a need to optimize for
    several objective functions. An approach is to decouple
    them. This makes model development and maintenance easier.
}.
% https://developers.google.com/machine-learning/problem-framing/problem-framing
At a high level, ML problem framing consists of two distinct steps: 
\begin{enumerate}
    \item Determining whether ML is the right approach for solving
    a problem.
    \item Framing the problem in ML terms.
\end{enumerate}
%% Designing MLS. Huyen, 2022.
If you can frame (a part of) the product goal as a
learning problem, we unlock many possible solutions.
For one product goal, there are usually many different
ML formulations, with varying levels of implementation
difficulty.



% https://developers.google.com/machine-learning/problem-framing/problem-framing
\subsection*{Understand the problem}
To understand the problem, perform the following tasks:
\begin{itemize}
    \item \textbf{State the goal} for the product you are developing
    or refactoring. Begin by stating your goal in non-ML terms.
    The goal is the answer to the question,
    "\textbf{What am I trying to accomplish?}"

    When we build a product or a service, we start by the product or
    service we want to deliver to users. On the other hand, ML
    problems are framed in terms of learning a relationship from data.


    \item \textbf{Evaluate Feasibility}, that is, determine if the
    goal is best met using Statistics/ML/AI, or a non-ML solution
    \footnote{
        Be open to any approach whether it requires ML or not.

        Some view ML as a universal tool that can be applied to
        all problems. In reality, ML is a specialized tool
        suitable only for particular problems. You don't want
        to implement a complex ML solution when a simpler non-ML
        solution will work.
    }.
    
    All ML problems are not created equal! As our understanding of ML
    has evolved, problems such as building a model to correctly
    classify photos of cats and dogs have become solvable in a matter
    of hours, while others, such as creating a system capable of
    carrying out a conversation, remain open research  problems
    \footnote{
        To efficiently build ML applications, it is
        important to consider multiple potential ML
        framings and start with the ones we judge as
        the simplest. One of the best ways to evaluate
        the difficulty of an ML problem is by looking at
        both the kind of data it requires and at the
        existing models that could leverage said data.
    }.


    \item \textbf{Verify you have the data required} to train a model
    if you're using a predictive ML approach.

    Think about which kind of data you have available to you or could
    gather. Oftentimes, data availability ends up being the limiting
    factor in model selection.
\end{itemize}
Your choice of model type depends upon the specific context and
constraints of your problem. The model's output should
(at least help to) accomplish the goal.  Determining the goal
and which kind of inputs your model will take in and which outputs
it will produce will help you narrow down potential approaches
significantly. Based on these types, any of the relevant categories
of ML approaches could be a good fit. 



\newpage
\subsection*{Evaluating Data Feasibility.}
To succesfully evaluate the feasibility of a ML project involves
several key steps:
\begin{enumerate}
    \item [1.] \textbf{Understanding the Problem}
    
    \noindent
    Define objectives, inputs, and outcomes.


    \item [2.] \textbf{Data Collection and Masterization}
    
    \noindent
    Identify \textbf{data sources}, required attributes, and datatypes.


    \item [3.] \textbf{Data Quality}
    
    \noindent
    Check if the data is:
    \begin{itemize}
        \item \textbf{complete} or if there are significant gaps; 
        
        \item \textbf{consistent and reliable}
        \footnote{
            e.g., a model will benefit from data gathered over
            many years from the same reliable instruments.
        }.
        Having data that's consistently and reliably collected will
        produce a better model. 
        
        \item \textbf{trusted}, that is, understand where your data
        will come from. Will the data be from trusted sources you trust
        \footnote{
            like logs from your product, or will it be from sources you
            don't have much insight into, like the output from another
            ML system?
        }?

        \item \textbf{available} at prediction time in the correct format.
        If it will be difficult to obtain certain feature values at
        prediction time, omit those features from your datasets.

        \item \textbf{accurate or correct}. In large datasets, it's
        inevitable that some labels will have incorrect values,
        but if more than a small percentage of labels are incorrect,
        the model will produce poor predictions.

        \item \textbf{representative} of the real world as possible.
        Training on unrepresentative datasets can cause poor performance
        when the model is asked to make real-world predictions.
    \end{itemize}


    \item [4.] \textbf{Availability or Abundance}
    
    \noindent

    Ensure there is sufficient historical data to train and
    validate the model. There are roughly three levels of data
    availability, from best-case scenario to most challenging.
    Unfortunately, as with most other tasks, you can generally
    assume that the most useful type of data will be the hardest to
    find.

    \begin{itemize}
        \item When working on a supervised model, finding a
        \textbf{labeled dataset} is every practitioner's dream
        \footnote{
            Finding a labeled dataset that fits your needs and is
            freely available on the web is rare in practice. It is
            common, however, to mistake the dataset that you find
            for the dataset that you need.
        }.

        \item Some datasets contain labels that are not exactly the
        modeling target, but somewhat correlated with it.
        
        Interaction history is an example of a \textbf{weakly
        labeled} dataset for predicting a user's preferences. Weak
        labels are less precise by definition but often easier to
        find than perfect labels.

        \item In some cases, while we do not have a labeled dataset
        mapping desired inputs to outputs, we at least have access
        to a dataset containing relevant examples.
        
        This means we need to label the dataset, find a model
        that can learn from unlabeled data, or do a little bit of
        both.

        \item In some cases we need to first \textbf{acquire it}.
    \end{itemize}

    \item[5.] \textbf{Data Processing} 
    
    \noindent
    Evaluate the \textbf{data processing needs},
    \textbf{pipeline feasibility}.

    \item[6.] \textbf{Identify Risks}
    
    \noindent
    Identify risks related to data, such as data breaches, changes
    in data availability, or quality issues. Develop strategies to
    mitigate these risks.

\end{enumerate}
Remember that datasets and models are iterative and start from
simple. Also, it may be a requirement to evaluate the latency or
ease of implementation.





% https://developers.google.com/machine-learning/problem-framing/problem-framing
\subsection{Clear use case for ML}
Based on the input and output, ML systems can be divided into
predictive and generative. 

\begin{table}[h!]
    \centering
    \begin{tabular}{|l|l|l|l|}
    \hline
    \textbf{Category} & \textbf{Input} & \textbf{Output} & \textbf{Training Technique} \\ \hline
    \textbf{Predictive ML} & \begin{tabular}[c]{@{}l@{}}Text\\ Image\\ Audio\\ Video\\ Numerical\end{tabular} & \begin{tabular}[c]{@{}l@{}}Makes a prediction, e.g., classifying an email as \\ spam or not spam, guessing tomorrow's rainfall,\\ or predicting the price of a stock. Output can be\\ verified against reality.\end{tabular} & \begin{tabular}[c]{@{}l@{}}Typically uses lots of data to train a supervised,\\ unsupervised, or reinforcement learning model\\ to perform a specific task.\end{tabular} \\ \hline
    \textbf{Generative AI} & \begin{tabular}[c]{@{}l@{}}Text\\ Image\\ Audio\\ Video\\ Numerical\end{tabular} & \begin{tabular}[c]{@{}l@{}}Generates output based on the user's intent, \\ e.g., summarizing an article, or producing an audio\\ clip or short video.\end{tabular} & \begin{tabular}[c]{@{}l@{}}Typically uses lots of unlabeled data to train a \\ large language model or image generator to fill\\ in missing data. Can be fine-tuned by training\\ on labeled data for specific tasks, like classification.\end{tabular} \\ \hline
    \end{tabular}
    \caption{Comparison of Predictive ML and Generative AI}
\end{table}

There's no value in predicting something if you can't turn the
prediction into an action that helps users
\footnote{
    For example, a model that predicts whether a user will find
    a video useful should feed into an app that recommends
    useful videos. A model that predicts whether it will rain
    should feed into a weather app.
}.
That is, your product should take action from the model's output.

Your choice of model type depends upon the specific context and
constraints of your problem. The model's output should accomplish
the task defined in the ideal outcome. Thus, the first question
to answer is "What type of output do I need to solve my problem?"

If you need to classify something or make a numeric prediction,
you'll probably use predictive ML. If you need to generate new
content or produce output related to natural language
understanding, you'll probably use generative AI. The following
tables list predictive ML and generative AI outputs:

\begin{table}[h!]
    \centering
    \begin{tabular}{|l|l|l|}
    \hline
    \textbf{Category} & \textbf{ML System} & \textbf{Example Output} \\ \hline
    \textbf{Classification} & Binary & Classify an email as spam or not spam. \\ \hline
    \textbf{} & Multiclass single-label & Classify an animal in an image. \\ \hline
    \textbf{} & Multiclass multi-label & Classify all the animals in an image. \\ \hline
    \textbf{Numerical} & Unidimensional regression & Predict the number of views a video will get. \\ \hline
    \textbf{} & Multidimensional regression & Predict blood pressure, heart rate, and cholesterol levels for an individual. \\ \hline
    \end{tabular}
    \caption{Predictive ML Examples}
\end{table}

\begin{table}[h!]
    \centering
    \begin{tabular}{|l|l|}
    \hline
    \textbf{Model Type} & \textbf{Example Output} \\ \hline
    \textbf{Text} & \begin{tabular}[c]{@{}l@{}}Summarize an article.\\ Reply to customer reviews.\\ Translate documents from English to Mandarin.\\ Write product descriptions.\\ Analyze legal documents.\end{tabular} \\ \hline
    \textbf{Image} & \begin{tabular}[c]{@{}l@{}}Produce marketing images.\\ Apply visual effects to photos.\\ Generate product design variations.\end{tabular} \\ \hline
    \textbf{Audio} & \begin{tabular}[c]{@{}l@{}}Generate dialogue in a specific accent.\\ Generate a short musical composition in a specific genre, like jazz.\end{tabular} \\ \hline
    \textbf{Video} & \begin{tabular}[c]{@{}l@{}}Generate realistic-looking videos.\\ Analyze video footage and apply visual effects.\end{tabular} \\ \hline
    \textbf{Multimodal} & Produce multiple types of output, like a video with text captions. \\ \hline
    \end{tabular}
    \caption{Generative AI Model Types and Example Outputs}
\end{table}





% Designing ML Sys
\subsection{Designing the ML System}
Usually, there are 4 main complementary requirements for the devs
of the ML system:
\begin{enumerate}
    \item \textbf{Reliability.}
    
    \noindent
    The system should continue to perform the correct function at 
    the desired level of performance even in the face of adversity
    (e.g. hardware, software or human faults).

    ML systems can fail silently and end users may not know it failed.
    This can be very dangerous.


    \item \textbf{Scalability.}
    
    \noindent
    This can mean that the ML system can handle growth in complexity,
    traffic volume, or model count. Whatever growth (or decrease)
    is expected,  there should be reasonable ways of dealing and
    managing the demands (e.g. a hundred models, where the
    training and monitoring need to me automated).


    \item \textbf{Maintainability.}
    
    \noindent
    Different contributors can work using tools that they are
    comfortable with instead of forcing tools onto other groups.

    Code and common bugs should be documented. Code, data, and 
    artifacts should be versioned.


    \item \textbf{Adaptability.}
    
    \noindent
    To distribution shifts and dynamic business requirements. 
    The system can be updated without service interruption.
\end{enumerate}





% Ameisen, 2020
\subsection{Framing the ML Editor}

\textbf{How could we iterate through a product use case to find the right ML framing?}
Since most product goals are very specific, attempting to
solve an entire use case by learning it from end-to-end often
requires custom-built cutting-edge ML models. This may be the
right solution for teams that have the resources to develop and
maintain such models, but it is often worth it to start with
more well-understood models first.

It is often a great idea to be the algorithm before they
implement it. In other words, to understand how to best
automate a problem, start by attempting to solve it manually.
So, how would we go about it?

If we start with a human heuristic and then build this simple
model, we will quickly be able to have an initial baseline,
and the first step toward a solution. Moreover, the
initial model will be a great way to inform what to build next.


% Ameisen, 2020
\subsection*{Monica Rogati: How to Choose and Prioritize ML Projects}
The ideal case is that you can pitch the results regardless of the outcome: if you do
not get the best outcome, is this still impactful? Have you learned something or validated
some assumptions? A way to help with this is to build infrastructure to help
lower the required effort for deployment.

Q: How do you scope out an ML product?

You have to remember that you are trying to use the best tools to
solve a problem, and only use ML if it makes sense.


Q: How do you decide what to focus on in an ML project?

You have to find the impact bottleneck, meaning the piece of your pipeline that
could provide the most value if you improve on it. When working with companies, I
often find that they may not be working on the right problem or not be at the right
growth stage for this.

There are often problems around the model. The best way to find this out is to
replace the model with something simple and debug the whole pipeline. Frequently,
the issues will not be with the accuracy of your model. Frequently, your product is
dead even if your model is successful.


Q: Why do you usually recommend starting with a simple model?

The goal of our plan should be to derisk our model somehow. The best way to do
this is to start with a “strawman baseline” to evaluate worst-case performance.

Looking at your data helps you think of good heuristics, models, and ways to reframe
the product.


\section{Create a Plan}
Many ML projects are doomed from the start due to a misalignment between product
metrics and model metrics. More projects fail by producing good models that
aren't helpful for a product rather than due to modeling difficulties.

\

\subsection{Measuring Success}
We have three potential approaches for increasing the complexity of ML models
\footnote{
    It is important to also realize that even features that could benefit from
    ML can often simply use a heuristic for their first version. Once the
    heuristic is being used, you may even realize that you do not need ML at all.
}:
\begin{enumerate}
    \item[1.] Baseline, design heuristics based on domain knowledge.
    \item[2.] Simple Model. 
    \item[3.] Complex Model.
\end{enumerate} 
All of these approaches are different and may evolve as we learn more from
prototypes along the way, but when working on ML, you should define a common
set of metrics to compare the success of modeling pipelines.

To that end, we will cover four categories of performance that have a
large impact on the usefulness of any ML product: business metrics,
model metrics, freshness, and speed. Clearly defining these metrics will
allow us to accurately measure the performance of each iteration.

\begin{itemize}
    \item \textbf{Business Performance}
    
    \noindent
    Once the product or feature goal is clear, a metric should be defined
    to judge its success. This metric should be separate from any model
    metrics and only be a reflection of the product's success
    \footnote{
        Product metrics may be as simple as the number of users a feature
        attracts or more nuanced such as the click-through rate (CTR) of
        the recommendations we provide.
    }.

    Product metrics are ultimately the only ones that matter, as they
    represent the goals of your product or feature. All other metrics
    should be used as tools to improve product metrics
    \footnote{
        Product metrics, however, do not need to be unique. While most
        projects tend to focus on improving one product metric, their
        impact is often measured in terms of multiple metrics,
        including guardrail metrics, metrics that shouldn't decline
        below a given point.
    }. 


    \item \textbf{Model Performance}
    
    \noindent
    The ultimate product metric that determines the success of a model
    is the proportion of visitors who use the output of a model out of
    all the visitors who could benefit from it.

    When a product is still being built and not deployed yet, it is not
    possible to measure usage metrics. To still measure progress, it is
    important to define a separate success metric called an offline
    metric or a model metric.

    Small changes to the interaction between the model and product can
    make it possible to use a more straightforward modeling approach
    and deliver results more reliably
    \footnote{e.g. Changing an interface so that a model's results can
    be omitted if they are below aconfidence threshold; Presenting a
    few other predictions or heuristics in addition to a model's top
    prediction; Communicating to users that a model is still in an
    experimental phase and giving them opportunities to provide feedback
    }.


    \item \textbf{Freshness and Distribution Shift}
    
    \noindent
    Even if a model is trained on an adequate dataset, many problems have
    a distribution of data that changes as time goes on. When the distribution
    of the data shifts, the model often needs to change as well in order to
    maintain the same level of performance. In general, a model can perform
    well on data it hasn't seen before as long as it is similar enough to
    the data it was exposed to during training.

    Not all problems have the same freshness requirements.

    Depending on your business problem, you should consider how hard it will
    be to keep models fresh. How often will you need to retrain models, and
    how much will it cost you each time we do so?


    \item \textbf{Speed}
    
    \noindent
    Ideally, a model should deliver a prediction quickly. This allows users
    to interact with it more easily and makes it easier to serve a model to
    many concurrent users. So how fast does a model need to be?


\end{itemize}



\subsection{Estimate Scope and Challenges}
In ML, success generally requires understanding the context of the task well,
acquiring a good dataset, and building an appropriate model. Let's elaborate.

\begin{itemize}
    \item \textbf{Leverage Domain Expertise}
    
    \noindent
    The simplest model we can start with is a heuristic: a good rule of thumb based on
    knowledge of the problem and the data. The best way to devise heuristics is to see
    what experts are currently doing. Most practical applications are not entirely novel.
    How do people currently solve the problem you are trying to solve?

    The second best way to devise heuristics is to look at your data. Based on your dataset,
    how would you solve this task if you were doing it manually?

    To identify good heuristics, I recommend either learning from experts in the field or
    getting familiar with the data.


    \item \textbf{Stand on the Shoulders of Giants}
    
    \noindent

    Have people solved similar problems? If so, the best way to get started is to understand
    and reproduce existing results. Look for public implementations either with
    similar models or similar datasets, or both.

\end{itemize}
Leveraging existing open code and datasets can help make
implementation faster. In the worst case, if none of the existing models performs well
on an open dataset, you now at least know that this project will require significant
modeling and/or data collection work.

If you have found an existing model that solves a similar task and managed to train it
on the dataset it was originally trained on, all that is left is to adapt it to your domain.
To do so, I recommend going through the following successive steps:
\begin{enumerate}
    \item[1.] Find a similar open source model, ideally paired with a dataset it was trained on,
    and attempt to reproduce the training results yourself.

    \item[2.] Once you have reproduced the results, find a dataset that is closer to your use
    case, and attempt to train the previous model on that dataset.

    \item[3.] Once you have integrated the dataset to the training code, it is time to judge how
    your model is doing using the metrics you defined and start iterating.
\end{enumerate}



\subsection{To Make Regular Progress: Start Simple}
It is worth repeating that much of the challenge in ML is similar to one
of the biggest challenges in software—resisting the urge to build pieces
that are not needed yet.

Many ML projects fail because they rely on an initial data acquisition and
model building plan and do not regularly evaluate and update this plan.

Because of the stochastic nature of ML, it is extremely hard to predict how
far a given dataset or model will get us.

For that reason, it is vital to start with the simplest model that could
address your requirements, build an end-to-end prototype including this
model, and judge its performance not simply in terms of optimization metrics
but in terms of your product goal.

\subsection*{Start with a Simple Pipeline}
To do this, we will need to build a pipeline that can take data in and
return results. For most ML problems, there are actually two separate
pipelines to consider:
\begin{enumerate}
    \item[i. Training] 
    
    A training pipeline ingests all of the labeled data you would like
    to train on and passes it to a model. It then trains said model on
    the dataset until it reaches satisfactory performance.
    
    Most often, a training pipeline is used to train multiple models
    and compare their performance on a held-out validation set.


    \item[ii. Inference/Prediction] 

    At a high level, an inference pipeline starts by accepting input
    data and preprocessing it.
    
    The preprocessing phase usually consists of multiple steps. Most
    commonly, these steps will include cleaning and validating the
    input, generating features a model needs, and formatting the data
    to a numerical representation appropriate for an ML model.
    
    Pipelines in more complex systems also often need to fetch
    additional information the model needs such as user features
    stored in a database.


    \item[iii. Evaluation]
    
    From the predictions, what can we improve?
\end{enumerate}
\section{MLOps}



\subsection{What Is MLOps?}
Why isn't machine learning 10X faster?
Most of the problem-building machine learning systems involve
everything surrounding machine learning modeling: data engineering,
data processing, problem feasibility, and business alignment.

The quicker the feedback loop (Kaizen)
\footnote{
    What is Kaizen? In Japanese, it means improvement.
    Using Kaizen as a software management philosophy originated in
    the Japanese automobile industry after World War II. It
    underlies many other techniques: Kanban, root cause analysis,
    five why's, and Six Sigma.
    
    To practice Kaizen, an accurate and realistic assessment of
    the world's state is necessary and pursues daily, incremental
    improvements in the pursuit of excellence.
}
the more time to focus on business problems.

The reason models are not moving into production is the impetus
for the emergence of MLOps as a critical industry standard.
MLOps shares a lineage with DevOps in that DevOps philosophically
demands automation. A common expression is if it is not automated,
it's broken. Similarly, with MLOps, there must not be components of
the system that have humans as the levers of the machine
\footnote{
    The history of automation shows that humans are the least
    valuable doing repetitive tasks but are the most valuable
    using technology as the architects and practitioners.
}.



\subsection{An MLOps Hierarchy of Needs}
An ML system is a software system, and software systems work
efficiently and reliably when DevOps and data engineering best
practices are in place.

So how could it be possible to deliver the true potential of
machine learning to an organization if DevOps' basic foundational
rules don't exist or data engineering is not fully automated?


The ML hierarchy of needs in the next list is not a definitive guide
but is an excellent place to start a discussion.
\begin{enumerate}
    \item One of the major things holding back machine learning
    projects is this necessary foundation of \textbf{DevOps}.

    The foundation of DevOps is continuous integration. Without
    automated testing, there is no way to move forward with DevOps.
    Continuous integration is relatively painless for a Python
    project with the modern tools available. The first step is to
    build a "scaffolding" for a Python project which consists of:
    a \textbf{MakeFile}, \textbf{Requirements(.txt)},
    \textbf{code.py},  \textbf{test\_code.py}, \textbf{environment}
    \footnote{
        we recommend using conda for managing the environment and
        generating the requirements file.
    }.

    One of the most straightforward ways to implement CI for this
    Python scaffolding project is with \textbf{GitHub Actions}.

    CD, i.e., automatically pushing the machine learning project
    into production, would be the next logical step. This step
    would involve deploying the code to a specific location using
    a continuous delivery process and IaC (Infrastructure as Code).


    \item After the DevOps foundation is complete, next is
    \textbf{data automation}, that is, to automate the flow of data
    \footnote{
        e.g. with Apache Airflow, AWS Data Pipeline, AWS Guidelines,
        SchemaChange.
    }.

    Some items to consider here are the data's size, the frequency
    at which the information is changed, and how clean the data is.


    \item After DevOps and automated data comes
    \textbf{platform automation}, that is to evaluate is how an
    organization can use high-level platforms to build ML solutions.
    
    An ML platform solves real-world repeatability, scale, and
    operationalization problems.


    \item Finally, sssuming all of these other layers are complete
    (DevOps, Data Automation, and Platform Automation),
    \textbf{ML automation}, or \textbf{MLOps}, is possible.

    One way to articulate these best practices is to consider that
    they create reproducible models with robust model packaging,
    validation, and deployment. In addition, these enhance the
    ability to explain and observe model performance.

    The MLOps feedback loop includes the following:
    \begin{itemize}
        \item \textbf{Create and retrain models with reusable ML
        Pipeline.}
        
        \noindent
        Creating a model just once isn't enough. The data can
        change, the customers can change, and the people making
        the models can change. The solution is to have reusable
        ML pipelines that are versioned.
        

        \item \textbf{Continuous Delivery of ML Models.}
        
        \noindent
        CD of ML Models is similar to CD of software. When all of
        the steps are automated, including the infrastructure,
        using IaC, the model is deployable at any time to a new
        environment, including production.


        \item \textbf{Audit trail for MLOps pipeline.}
        
        \noindent
        It is critical to have auditing for machine learning models.
        There is no shortage of problems in machine learning,
        including security, bias, and accuracy. Therefore,
        having a helpful audit trail is invaluable, just as having
        adequate logging is critical in production software engineering
        projects. In addition, the audit trail is part of the
        feedback loop where you continuously improve your approach
        to the problem and the actual problem.
        

        \item \textbf{Monitor the model to improve future models.}
        
        \noindent
        One of the unique aspects of machine learning is that the
        data can literally "shift" beneath the model.

    \end{itemize}



\end{enumerate}
The culmination of MLOps is a machine learning system that works.
The people that work on operationalizing and building machine
learning applications are machine learning engineers and/or data
engineers.








\end{document}



% Emmanuel Ameisen, 2020. Building Machine Learning Powered Applications Going from Idea to Product
